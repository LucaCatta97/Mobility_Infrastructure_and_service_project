\chapter{Conclusion}

%Appunti da audio di andre

%- la prima parte è standard per tutti, quindi nelle conclusioni focalizziamoci sulla seconda

%- quelle robe qui le abbiamo ideate noi, ma dato che sono innovative non abbiamo un riscontro per capire se sono giuste o sbagliate, il nostro risultato si basa su decisioni che abbiamo preso (tipo che argomenti trattare) 

%- gli arfomenti sono innovativi ADESSO, ma tra 10 anni si spera che queste tecnolgie saranno la nuova normalita e son cose che andranno sempre aggiornate (di contratto in contratto ecco), se noi lo facciamo oggi e tra pochi anni c'è una orba rivoluzionaria di cui serve analizzare roba, questa dashboard ormai è obsoleta

%- considerare come si evolve il target di persone che utilizzano i mezzi: ora i mezzi ci sono ma tanti non li prendono, in un futuro quando si spera che diventi lo standard, il servizio cambiarà a seconda del target di persone che lo usano
%\newpage
%\subsection{conclusione fatta bene}

Initially, the aim of the project was to design a dashboard with KPIs describing the company's performance, making best use of the data they provided. However, we wanted to broaden the work, \emph{going beyond what Arriva has}, by starting to build hypotheses on \textit{what the company could have} and should take into account in the near future. The aim was mainly to provide the company with basic tools from which to develop a broader and more detailed work in this regard. Since we are familiar with the trends in this sector and, more generally, with what mobility aims to be in the future, we wanted to focus on two main topics: \textbf{new technologies} and \textbf{sustainability}, broken down into green and social.
In our opinion, these are the most important topics on which a public transport company such as Arriva must be sure to be well prepared, so that it can best manage the competition, which is becoming more and more intense among competitors nationally and internationally.

One of the biggest doubts we always asked ourselves during the creation of the Innovative part was whether we were actually doing a good job, in fact in the Traditional Dashboard the reconfirmation of our good work was given to us by the fact that the pages we composed were - \textit{de facto} - created following the requirements, penalties and rewards listed in the Service Contract between Arriva and the city of Bergamo. In the innovative part, on the other hand, our result was based on assumptions, hypotheses or particular literature research on what could be the best topics to cover and which KPIs to use.

Lastly, we would like to emphasize that although the topics covered are somewhat innovative, \textit{it is not certain that they will be so in 5 or 10 years' time}. Our job has been to initiate a new philosophy of thinking within the company, which then has the task of continuing to update and not set back and relax on what it has, knowing that technology advances as do the goals for a sustainable world. Moreover, the world of public transport, and of mobility in general, is undergoing exponential development that will lead to the discovery of new horizons: in the future, it is likely that the target users to which PTOs are currently aiming will change radically, forcing them to use new methods and new principles to best meet the new needs and behaviour of users.


