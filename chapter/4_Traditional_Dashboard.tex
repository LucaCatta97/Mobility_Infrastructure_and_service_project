\chapter{Analysis on the provided data}
\label{ch:dashboard}
%The dashboard created on PowerBI 
An off-line dashboard with PowerBI is created to discover useful information that without visualization could go lost in the background.

\section{Data cleaning}
Before importing the data, a cleaning of the provided dataset must be made to adapt it for the final purpose. 

The main problem is the absence of a field for dates. For example, data about kilometers run by each member of the consortium were provided as an excel file for each year considered and each file, in turn, was divided in 12 worksheet, one for each operative month. PowerBI needs a reference to join data and to create relationship between different data set, and for this reason, a key about date was created.  


\section{ER model}
An Entity–Relationship model it's been created, to set the relationships between the different tables, as shown in the figure (\ref{fig:ER}).

In detail, the table \textit{'Companies'} represents the monthly journeys provided by the Consortium, divided for each companies, from which data about commercial speed and number of rides traveled are extracted for each line. 

The table \textit{'Suppressed rides'} is related with \textit{'Companies'} through the companies' name. This table gives several information about the events related to the suppression of a ride; in particular between years 2017 and 2019, it declares the reason for why the ride was suppressed, the line and the company associated to that ride. 

\textit{'Multi\_Year'} is a table related with \textit{'Companies'} through line code and it shows introits and costs per kilometer for each line. 

\textit{'Flotta'} is a table without any relationship and it shows the data about Consortium's fleet for each year between 2018 and 2020 in terms of: emissions and kind of fuel used by each bus, if the bus has air condition, if the bus is equipped for transporting disable people and the age of each bus.


\begin{figure}[h]
    \centering
    \includegraphics[width=0.9\textwidth]{Images/traditional_dashboard/ER_model.png}
    \caption{ER model of the Dashboard}
    \label{fig:ER}
\end{figure}
\section{Dashboard}

Starting from these tables, a dashboard is designed. It will be divided in subcategories that differ each other for the topic shown. In particular, the themes approached are:
\begin{itemize}
\item Suppressed rides
\item Bus requirements
\item Operative performances
\item KPIs correlation
\item Economics
\end{itemize}
\subsection{Suppressed rides}
In figure \ref{fig:sup} has been presented the dashboard.
\newpage
\WarningsOff
\begin{landscape}
\thispagestyle{empty}
\includepdf[angle=90, pagecommand={\null\enlargethispage{2\baselineskip}\vfill\captionof{figure}{Suppressed}\label{fig:sup}}]{dashboard/Suppression.pdf}
\end{landscape}
\newpage
As said before, data about suppressed rides are between 2017 and 2019. An area of this part of the page is dedicated to filters, thanks to which data can be visualized according to:
\begin{itemize}
\item  period of time with a level of detail equal to a month
\item line
\item reason for which the ride was suppressed
\end{itemize}
Obviously, these filters could be used at the same time in order to obtain very specific analysis.

The table on the bottom right shows the number of rides traveled per each ride suppressed for each company. So, a low number corresponds to a bad result for the company. This data is computed in a relative way in order to show if a company has bad performances according to its dimension. A ride suppressed for a small company is not comparable with few rides suppressed for a big one. 

The pie chart has the goal of showing how the different causes affects suppressed rides, while the histogram shows the total suppressed rides for each year divided by company. 

Considering the dashboard without applying any filter, some considerations can be already done. In 2017 Sab/ai had a huge issue with suppressed rides that reached a percentage level double than the second highest one, in the analyzed period. In 2017 and 2018 Sai and Zani had no problems and their suppressed rides were equal to zero. Finally, Agi had a ever increasing problem related to suppressed rides. 

Moreover, the main cause that brings to a suppressed ride is strike followed at a distance by failure happened during service and inconvenience due to drivers during service. 

Finally, companies with performances below Consortium average are Agi and Sab/ai with, respectively, a ride suppressed every 2,118 and 1,528 rides traveled. 

\newpage

\begin{landscape}
\thispagestyle{empty}
\includepdf[angle=90,  pagecommand={\null\enlargethispage{2\baselineskip}\vfill\captionof{figure}{Suppressed for strikes}\label{fig:strikes}}]{dashboard/Suppression_for_strikes.pdf}
\end{landscape}
\newpage

A focus on the strikes, shown in figure \ref{fig:strikes}, gives several information. Strikes are the reason of the abnormal problems tha Sab/ai had in 2017, moreover a strike in 2019 hit almost every company in an equal measure and, finally, Agi growing issues during this period are reflected also in strikes trend. 

Filtering data according to failure happened during service, a very interesting result is shown: half of the total failures happened during the three year in the Consortium happened during a ride performed by Agi even if Agi is one of the companies with less kilometers run per year. Moreover, this problem in absolute terms remains constant during the year and the negative trend of Agi is due to the increasing of other issues that in 2017 were almost absent.

\newpage
\begin{landscape}
\thispagestyle{empty}
\includepdf[angle=90, pagecommand={\null\enlargethispage{2\baselineskip}\vfill\captionof{figure}{Focus on Agi and fault on the line}\label{fig:failline}}]{dashboard/Suppression_for_fail_on_the_line.pdf}
\end{landscape}
\newpage

\subsection{Fleet}

The second part of the dashboard regards an overview of the fleet based on the aspect that are underlined in the contract of service and that should be considered as constraint for a good fleet management, even if not all of them brings directly to a penalty from the Public Transport Authority. 

In particular, in the contract the PTA requires the following fleet criteria:
\begin{itemize}
\item $100\%$ buses must be air-conditioned
\item $91\%$ accessible for people with reduced mobility
\item  $100\%$ eco-diesel
\item at least $62\%$ Euro V, Euro VI, EEV or technologies with better emission performances.
\item  age restrictions:
    \begin{itemize}
        \item No restrictions if the age is lower of 18 years
        \item Travel limitation between 19 and 21 years, they can't be over $20\%$ of the total fleet
        \item Not allowed if they are over 22 years
    \end{itemize}
\end{itemize}
The figure \ref{fig:fleet}
shows the fleet situation in 2020, that is the most recent situation available. Data about 2018 and 2019 fleet are also available, so that comparison between these years can be done. Each aspect listed before is summed up in a pie chart that let a clear visualization of the situation.

Beginning from the good aspects, the entire fleet uses an eco-diesel fuel and almost every bus is equipped to allow the transport of people with reduced mobility. 

In Italy the average fleet age is about 12 years\cite{rossiPTM}, but in 2020 the Consortium results are even better with an average age of about 10 years. Despite this, 10 buses have an age that obliges them to suffer travel limitations and the $13\%$ in the immediate future will be in this situation. 

On the other side, as much as 8 buses are not equipped with air conditioning and, according to the contract of service, they should not travel. Finally, it is possible analyze the fleet according to their emission category. The requirements set by the contract ($62\%$) are widely satisfied, but the improvement margin is really high since the $30\%$ of the fleet has an high polluting engine. 


\newpage
\begin{landscape}
\thispagestyle{empty}
\includepdf[angle=90, pagecommand={\null\enlargethispage{2\baselineskip}\vfill\captionof{figure}{Fleet dashboard}\label{fig:fleet}}]{dashboard/emission.pdf}
\end{landscape}
\newpage

In the figure \ref{fig:fleetold}  a focus on the oldest part of the fleet in 2020, the one that has constraint on traveled kilometers. 

It can be seen that the majority of those buses are also the ones that are not air-conditioned, so, if they are replaced with new ones, two bad situations can be improved . Moreover, two of the three buses not equipped for people with reduced mobility belongs to this category. Unfortunately, if they are removed, an important percentage of not polluting buses would be lost. In fact, the $5\%$ of the fleet is composed of old buses with a swapped engine in order to respect contract limitations. Finally, the presence of buses with an age over 17 years is really increased respect to 2018, with a rise of over $200\%$.


The third dashboard (figure \ref{fig:fleetinquinanti}) is a focus on the most polluting part of the 2020 fleet, so buses with Euro 3 and Euro 4 standard technology. Fortunately, the majority of them has the less polluting engine, symptom of good turnover of the fleet during the previously years. Theoretically, their presence in the fleet is not against the rules written in the Service Contract, but from a modern point of view, they could be a stain in terms of sustainability and green attitude. In particular, in this dashboard an analysis about the newest subpart of fleet, the buses under 15 years, is performed. This focus is motivated by the fact that, despite their polluting nature, according to the contract, these buses could travel for many years (at least 6) on Bergamo province streets. So, these buses are all air-conditioned and designed for people with reduced mobility. Despite the set constraint of an age under 15 years, their average age is really high, 14 years, showing that their matriculation happened in a small period of time. Moreover, their decrease is consistent and regular in the last years and this shows a huge presence among buses that have not yet contractual restrictions. This last consideration is confirmed by the previous dashboard. Since this part of the fleet has several years in front of it and since its characteristics respect all the contract requirements, a suggestion could be an engine swapping with a less polluting standard then an expensive turnover of the entire fleet, or better, a mix of the two solutions.
\newpage
\begin{landscape}
\thispagestyle{empty}
\includepdf[angle=90, pagecommand={\null\enlargethispage{2\baselineskip}\vfill\captionof{figure}{Focus on old fleet}\label{fig:fleetinquinanti}}]{dashboard/fleet_old_2020.pdf}
\end{landscape}
\newpage


\newpage
\begin{landscape}
\thispagestyle{empty}
\includepdf[angle=90, pagecommand={\null\enlargethispage{2\baselineskip}\vfill\captionof{figure}{Focus on old fleet}\label{fig:fleetold}}]{dashboard/fleet_inquinanti.pdf}
\end{landscape}
\newpage

\subsection{Operative performances}
The aim of this page is to present the ratio of rides made by the companies of the consortium. 

In the bottom part of the figure \ref{fig:op} there are two filters that allow the selection of Lines and Years.

Above them,  the average of the percentage of rides made by the companies\footnote{the reader can argue that some of the companies has a percentage of ride made up to $100\%$ but they had suppression in the previous page. This is due to the fact that the quantity of suppression is very low with respect to the rides made each month by the companies. For this reason in the data can have been approximated}. Moreover, these data include 2020 and 2021, so companies with good performances before could have suffered diseases due to Covid-19. On the right, the commercial speed divided by lines is shown.

Finally, on the top the correlation between the average percentage of rides made and the commercial speed per companies.

With only this view can been seen four cluster of companies:
\begin{itemize}
    \item \textbf{sai} and \textbf{tbso} that has less than 28 km/h
    \item \textbf{locatelli} near 30 km/h
    \item \textbf{arriva}, \textbf{zani} around 35 km/h
    \item \textbf{agi} that are above 38 km/h 
\end{itemize}
This result, if combined with the timetable of the service (not provided) can help to find the company with the better performances. With the information provided can not be sure that agi is the member with better performance because it depends on the route that agi buses travel and on the time in which it operates. 



\newpage
\begin{landscape}
\thispagestyle{empty}
\includepdf[angle=90, pagecommand={\null\enlargethispage{5\baselineskip}\vfill\captionof{figure}{Operativ performances}\label{fig:op}}]{dashboard/operative_performances.pdf}
\end{landscape}
\newpage

\subsection{KPIs Correlation}
The page represented in \ref{fig:corr} contains the relationships between:
\begin{itemize}
  \item the cost and the income
   \item the ratio I/C and the the commercial speed
    \item ratio I/C and the suppression
\end{itemize}
From cost-income graph, it is clear that Tbso has the worst economic situation because it spends a considerable amount of money per km, even if their introits are the lowest ones. For a better clearness, the median of costs and incomes is drawn.

For the other two graphs, the ratio between income and cost is set as symbol of efficiency. With this hypothesis, it is possible to plot the relationship with suppressed rides and commercial speed in order to see if these two factors have influences on the performances of the companies. 

The graph with the commercial speed as x-axis is emblematic. Excluding Sai that seems to be an outlier for its very good efficiency, despite its low commercial speed, for the other companies the correlation is striking: higher the speed is, better the economic efficiency is. 

On the other side, the same thing can not be said for the suppressed lines. They do not seem to cause any issue to economic performance, indeed the companies that suffer more this problem are also the most efficient.

Unfortunately, this kind of dashboard is difficult to deepen if this characteristics are maintained. It could be more interesting evaluating the single lines, since they are more influenced on aspect such as commercial speed. From figure \ref{fig:corrlines} for example, it is evident that not all the lines traveled by Tbso have high cost, but only V10. This should lead the company to work on understanding the problems that this line creates. On the other side, line V20, always under Tbso management, has incredibly bad performances about incomes. 

The two lines that indicate the median cost and the median income become useful to divide the lines in four categories:
\begin{enumerate}
\item The best: high income, low cost
\item The satisfactory: high income, high cost
\item The improvable: low income, low cost
\item The worst: low income, high cost
\end{enumerate}
The lines that belong to the last category are the aforementioned line V10 and line F (Bergamo - Treviglio). 
On the other side, the best performances are obtained by line V, and this is curious because its underlines V10 and V20 have huge problems, line R (Bergamo - Soncino) and line U (Bergamo - Treviglio). 

It is interesting that the correlation between commercial speed and economic efficiency disappears when the situation is analyzed considering the single lines. Probably, since some lines are managed by more than one company, the society that controls the service has a big impact on the economic performances.

Focusing on 2020, a disastrous year for the transport sector, it is possible to notice that the efficiency of each line is stable, without any substantial difference determined by commercial speed. This could be a symptom of Covid-19 that hit equally every companies. The graph on the right shows that the median of incomes slightly decreased, while the median of costs considerably increased respect to the three-years period 2017/19 of the previous dashboard. In order to have a clearer idea of the performances, a comparison of the selected year with the two years before can be seen. In 2020, the disaster is evident. The costs grew by 13$\%$ respect 2019, probably because the higher incidence of fixed costs, but the dramatic data is about incomes that dropped by 35$\%$. Consequently, the efficiency decreased by 42$\%$. An interesting aspect is that the commercial speed did not rise hugely, despite of the lockdown period in which streets were empty. 

\newpage
\begin{landscape}
\thispagestyle{empty}
\includepdf[angle=90, pagecommand={\null\enlargethispage{3\baselineskip}\vfill\captionof{figure}{Page about correlation}\label{fig:corr}}]{dashboard/correlation.pdf}
\end{landscape}
\newpage


\newpage
\begin{landscape}
\thispagestyle{empty}
\includepdf[angle=90, pagecommand={\null\enlargethispage{3\baselineskip}\vfill\captionof{figure}{Percentage difference}\label{fig:corrlines}}]{dashboard/correlation_2.pdf}
\end{landscape}
\newpage
\begin{landscape}
\thispagestyle{empty}
\includepdf[angle=90, pagecommand={\null\enlargethispage{3\baselineskip}\vfill\captionof{figure}{Percentage difference}\label{fig:corrlines2}}]{dashboard/correlation_without_year.pdf}
\end{landscape}

\subsection{Income}
The last part of the dashboard is about a large picture of the economic situation. The screenshot that is shown at the following page \ref{fig:income} is one of the many example in which this kind of graph could be shown. It shows the economic efficiency of the entire Consortium in the last 10 years, but then it focuses on the performances of a set of lines and finally it shows their situation in the desired period. In the example, a spotlight on Arriva situation. Arriva manages three lines: M (Bergamo - Crema), Q (Bergamo - Chiari) and R (Bergamo - Soncino). Two of them, M and R have performances above the average, while line Q has serious problems in economic efficiency. Finally, a focus on line M shows that the economic situation on that line is slowly improving. 

On the right side of the page, two graphs about incomes and costs per kilometer in the last decade. In particular the ones in the figure \ref{fig:income} refer to M line. The graphs evidence that the increasing economic efficiency of this line happened because of a constant incomes growth, while, on the other hand, the costs were stable with a drop in 2014. The 2020 disaster is confirmed once more by the dramatic decrease of incomes and the consistent increase of costs.
\newpage
\begin{landscape}
\thispagestyle{empty}
\includepdf[angle=90, pagecommand={\null\enlargethispage{2.5\baselineskip}\vfill\captionof{figure}{Percentage difference}\label{fig:income}}]{dashboard/income.pdf}
\end{landscape}
\WarningsOn