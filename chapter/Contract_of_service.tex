\section{Analysis of Contract of Service}
%Analysis of the contract of service. From that  the defined KPI  to put in "traditional dashboard"
\paragraph{What is a service contract?}
%explain what is the service contract and some background in the context
The service contract regulates the discharge of a public service by an operator as decided by the public autorithy. In case of public transport the autorithy is call PTA and the operator PTO. The service contract are generally made after a tender and are regulated by the REGULATION (EC) No 1370/2007 at european level and by the italian legislative decree No 422/2007 that respeclty define the norms about the public transport service on road and rails and assign the function to the different italian local entities.

\paragraph{Our study case: Bergamo Trasporti Consortium}
The study case is about the public transport service of the subnetwork sud of the province of Bergamo (with the Province of Bergamo as PTA) and the Bergamo trasporti consortium as PTO.

\subparagraph{History of the service} The service has started in 2005 of the service after the winning of the tender by the consurtion in 2004. Then a first extension has been made in 2011 (the natural expiration year) until 2014 without changes in the requirements. Then another extension has been made with changes in the contract of service in particular:
\begin{itemize}
    \item the required number of $bus-km\cdot year$ rised up to $4345000$
    \item the grants
    \item the bus fleet
\end{itemize}

Then in 2019 the PTA use the power provided by the article 5 of REGULATION (EC) No 1370/2007 and use the \emph{requriment to provide} to extend the duration of the contract of service until 31/12/2021. This act also change the previous requirements/conditions:
\begin{itemize}
    \item 4.240.000 vetture-km
\end{itemize}

\subparagraph{Effects of COVID-19 in the contract of service}
Due to the COVID-19 emergency the PTA and PTO make a change in the contract of service in particular in the authorization of changes in service due to COVID prevention measure (example school closure) and the change of the bus fleet requirements with an increase of the maximum age of the bus from 15 years to 18 years without restriction and allow the bus with a maxium age of 21 years if the not exced the maxium of 1 milion km.

Then in 2021 with the last requirement ot provide the service has been extended until 2023.

\subparagraph{KPIs from the Contract of Service}
From the contract of service can be directly obtained the following KPIs\footnote{For better visualizing the trend a focus on the last requirement to provide has been made to have a more actual legislative situation}:
\begin{itemize}
    \item Bus fleet requirements:
    \begin{itemize}
        \item Age of the fleet (the last requirement to provide put a maximum of 18 years and 22 year 21 with restrinctions)
        \item If it is climatezed (all the fleet must be climatized)
        \item The emissions regulation (EU6, EU5, EU4, ...).
        \item the accessibility for people no autosufficiently
    \end{itemize}
    \item suppressed lines with the distiction by lines, years and companies
\end{itemize}

