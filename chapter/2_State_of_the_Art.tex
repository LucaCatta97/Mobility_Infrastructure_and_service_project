\chapter{State of the Art}

\paragraph{What is Arriva}
Arriva Italia is the Arriva Group’s italian branch. Present on the Italian market since 2002, managing around $5\%$ of the market shares, it provides both urban and extra-urban transport services mainly in Northern Italy, as well as shuttle services to Turin and Milan airports.

With respect to our case study Arriva Italia is member of the Bergamo Trasporti consortium from the 2002 with the acquisition of \emph{SAB Autoservizi} and directly from 2020 with the incorporation of SAB.

\subparagraph{The Consortium}
The Consortium is an association of different companies built in 2003 for the tender about the Subnetwork SUD of the Province of Bergamo.
The different companies that are part of the consortium are: \emph{SAI Società Autolinee Interprovinciali srl, Arriva Italia srl, AGI Auto Guidovie Italiane SpA, Autoservizi Locatelli srl, TBSO Trasporti Bergamo Sud Ovest SpA e Autoservizi Zani srl.}

\paragraph{What is a dashboard}
As reported on the official website of Microsoft a dashboard is a tool to track, analyze and display data about a process or an organization to obtain insight.

The benefits are different such as performance measure, data transparency and forecasting.


\subparagraph{Apllication on the LPT\cite{rossiPTM}}

In our study case the dashboard can be a useful instrument to allow both the PTO and PTA check the respect of the requirements provided from the contract of service. Then the PTO can define useful relation between the KPIs themselves. 

So, to start our analysis the Service contract must be read and analyzed.