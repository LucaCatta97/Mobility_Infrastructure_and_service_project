\chapter{Innovative KPIs}
\label{ch:Innovative}
\section{Introduction on the Innovative KPIs}
\label{sec:Intro_Innvovative}
In the last few years there is a growing interest in the concept of sustainability and sustainable transportation. New technologies and new objective to be achieved make it necessary to define new indicators to measure new relevant aspects related to sustainability.

The fundamental role of sustainable development is also confirmed by the 2030 Agenda for Sustainable Development, adopted by all United Nations Member States in 2015. It provides a shared blueprint for peace and prosperity for people and the planet, now and into the future. 

The 2030 Agenda defines \emph{17 Sustainable Development Goals (SDGs)}\cite{sdgs} referred to different areas of social, economic and environmental development and each goal has specific objectives to be achieved over the next few years. 

In the following table the 17 SDGs are reported the ONU target in 2030 and the respective sustainability topics that can be faced in a public transport company.

\newpage
\newpage

\begin{minipage}[c]{0.2\textwidth}
\centering
SDGs ONU
\end{minipage}
\fbox{
\begin{minipage}[c]{0.5\textwidth}
\centering
Target ONU 2030
\end{minipage}
}
\begin{minipage}[c]{0.2\textwidth}
\centering
LPT 
\end{minipage}


\begin{minipage}[c]{0.2\textwidth}
    \includegraphics[width=\textwidth]{Images/Social_sustainability/1_no_poverty.png}
\end{minipage}
\fbox{
\begin{minipage}[c]{0.5\textwidth}
1.3 Apply at national level adequate systems and social protection measures for all, including minimum levels, and by 2030 achieve substantial coverage of the poor and vulnerable
\end{minipage}
}
\begin{minipage}[c]{0.2\textwidth}
\emph{Corporate welfare}
\end{minipage}


\begin{minipage}[c]{0.2\textwidth}
    \includegraphics[width=\textwidth]{Images/Social_sustainability/3_helth.png}
\end{minipage}
\fbox{
\begin{minipage}[c]{0.5\textwidth}
3.8 Achieve universal health coverage, including protection from financial risks, access to essential quality health care services and access to safe, effective, quality and affordable essential drugs and vaccines for all
\end{minipage}
}
\begin{minipage}[c]{0.2\textwidth}
\emph{Corporate welfare}
\end{minipage}

\begin{minipage}[c]{0.2\textwidth}
    \includegraphics[width=\textwidth]{Images/Social_sustainability/white_box.png}
\end{minipage}
\fbox{
\begin{minipage}[c]{0.5\textwidth}
3.6 By 2020, halve the number of deaths and injuries from road traffic accidents worldwide
\end{minipage}
}
\begin{minipage}[c]{0.2\textwidth}
\emph{Health and safety}
\end{minipage}

\begin{minipage}[c]{0.2\textwidth}
    \includegraphics[width=\textwidth]{Images/Social_sustainability/4_education.png}
\end{minipage}
\fbox{
\begin{minipage}[c]{0.5\textwidth}
4.3 By 2030, ensure equal access for all women and men to affordable, high-quality education, vocational and third-level education, including university

4.4 By 2030, substantially increase the number of young people and adults who have the necessary skills, including technical and professional skills, for employment, decent work and entrepreneurial capacity

4.7 By 2030, ensure that all students acquire the knowledge and skills necessary to promote sustainable development through, inter alia, education for sustainable development and sustainable lifestyles, human rights, equality of gender, the promotion of a culture of peace and non-violence, global citizenship and the enhancement of cultural diversity and the contribution of culture to sustainable development
\end{minipage}
}
\begin{minipage}[c]{0.2\textwidth}
\emph{Training and\\ development}
\end{minipage}
\newpage
\begin{minipage}[c]{0.2\textwidth}
\centering
SDGs ONU
\end{minipage}
\fbox{
\begin{minipage}[c]{0.5\textwidth}
\centering
Target ONU 2030
\end{minipage}
}
\begin{minipage}[c]{0.2\textwidth}
\centering
LPT 
\end{minipage}


\begin{minipage}[c]{0.2\textwidth}
    \includegraphics[width=\textwidth]{Images/Social_sustainability/5_gender.png}
\end{minipage}
\fbox{
\begin{minipage}[c]{0.5\textwidth}
5.1 End all forms of discrimination against all women, girls and boys
from all over the world

\end{minipage}
}
\begin{minipage}[c]{0.2\textwidth}
\emph{Training and\\ development}
\end{minipage}

\begin{minipage}[c]{0.2\textwidth}
    \includegraphics[width=\textwidth]{Images/Social_sustainability/white_box.png}
\end{minipage}
\fbox{
\begin{minipage}[c]{0.5\textwidth}
5.5 Guarantee women full and effective participation and equal leadership opportunities at all levels of decision-making in political, economic and public life
\end{minipage}
}
\begin{minipage}[c]{0.2\textwidth}
\emph{Merit\\ management \\of employee}
\end{minipage}

\begin{minipage}[c]{0.2\textwidth}
    \includegraphics[width=\textwidth]{Images/Social_sustainability/6_water.png}
\end{minipage}
\fbox{
\begin{minipage}[c]{0.5\textwidth}
6.3 By 2030, improve water quality by reducing pollution, eliminating uncontrolled discharge practices and minimizing the release of hazardous chemicals and materials, halve the percentage of untreated wastewater and substantially increase recycling and safe reuse globally

\end{minipage}
}
\begin{minipage}[c]{0.2\textwidth}
\emph{Waste}
\end{minipage}

\begin{minipage}[c]{0.2\textwidth}
    \includegraphics[width=\textwidth]{Images/Social_sustainability/white_box.png}
\end{minipage}
\fbox{
\begin{minipage}[c]{0.5\textwidth}
6.4 By 2030, substantially increase water efficiency to be used in all sectors and ensure freshwater withdrawals and supplies to address water scarcity and substantially reduce the number of people suffering from water scarcity
\end{minipage}
}
\begin{minipage}[c]{0.2\textwidth}
\emph{Water \\consumption}
\end{minipage}


\begin{minipage}[c]{0.2\textwidth}
    \includegraphics[width=\textwidth]{Images/Social_sustainability/7_energy.png}
\end{minipage}
\fbox{
\begin{minipage}[c]{0.5\textwidth}
7.2 By 2030, significantly increase the share of renewables in the global energy mix

7.3 By 2030, double the global rate of energy efficiency improvement
\end{minipage}
}
\begin{minipage}[c]{0.2\textwidth}
\emph{Energetic\\ consumption}
\end{minipage}

\newpage

\begin{minipage}[c]{0.2\textwidth}
    \includegraphics[width=\textwidth]{Images/Social_sustainability/8_work.png}
\end{minipage}
\fbox{
\begin{minipage}[c]{0.5\textwidth}
8.1 Supporting per capita economic growth according to national circumstances and, in particular, at least 7 percent annual growth of gross domestic product in least developed countries
\end{minipage}
}
\begin{minipage}[c]{0.2\textwidth}
\emph{Production and distribution of economic value}
\end{minipage}

\begin{minipage}[c]{0.2\textwidth}
    \includegraphics[width=\textwidth]{Images/Social_sustainability/white_box.png}
\end{minipage}
\fbox{
\begin{minipage}[c]{0.5\textwidth}
8.2 Achieve higher levels of economic productivity through diversification, technological updating and innovation, including through a focus on high value-added sectors and labour-intensive sectors
\end{minipage}
}
\begin{minipage}[c]{0.2\textwidth}
\emph{Quality of service, customer satisfaction}
\end{minipage}


\begin{minipage}[c]{0.2\textwidth}
    \includegraphics[width=\textwidth]{Images/Social_sustainability/white_box.png}
\end{minipage}
\fbox{
\begin{minipage}[c]{0.5\textwidth}
8.4 Progressively improve, until 2030, the efficiency of global resources in consumption and production in an attempt to decouple economic growth from environmental degradation, in accordance with the ten-year framework of programs on sustainable consumption and production, with developed countries taking the initiative
\end{minipage}
}
\begin{minipage}[c]{0.2\textwidth}
\emph{Supply-chain management}
\end{minipage}

\begin{minipage}[c]{0.2\textwidth}
    \includegraphics[width=\textwidth]{Images/Social_sustainability/white_box.png}
\end{minipage}
\fbox{
\begin{minipage}[c]{0.5\textwidth}
8.5 By 2030, achieve full and productive employment and decent work for all women and men, including young people and people with disabilities, and equal pay for work of equal value

8.6 By 2020, substantially reduce the percentage of unemployed young people who do not follow a course of study or who do not follow training courses
\end{minipage}
}
\begin{minipage}[c]{0.2\textwidth}
\emph{Corporate welfare}
\end{minipage}

\begin{minipage}[c]{0.2\textwidth}
    \includegraphics[width=\textwidth]{Images/Social_sustainability/white_box.png}
\end{minipage}
\fbox{
\begin{minipage}[c]{0.5\textwidth}
8.8 Protect labour rights and promote a safe and secure working environment for all workers, including migrant workers, especially migrant women, and those in precarious work
\end{minipage}
}
\begin{minipage}[c]{0.2\textwidth}
\emph{Health and safety of workers}
\end{minipage}

\newpage

\begin{minipage}[c]{0.2\textwidth}
    \includegraphics[width=\textwidth]{Images/Social_sustainability/9_industry.png}
\end{minipage}
\fbox{
\begin{minipage}[c]{0.5\textwidth}
9.1 Develop quality, reliable, sustainable and resilient infrastructure, including regional and cross-border infrastructure, to support economic development and human well-being, with a focus on equitable access for all
\end{minipage}
}
\begin{minipage}[c]{0.2\textwidth}
\emph{Quality of service, accessibility, customer satisfaction}
\end{minipage}

\begin{minipage}[c]{0.2\textwidth}
    \includegraphics[width=\textwidth]{Images/Social_sustainability/white_box.png}
\end{minipage}
\fbox{
\begin{minipage}[c]{0.5\textwidth}
9.5 Strengthen scientific research, promote the technological capacities of industrial sectors in all countries, in particular in developing countries, including by encouraging innovation by 2030 and by substantially increasing the number of workers in the research and development every million people and public and private spending on research and development
\end{minipage}
}
\begin{minipage}[c]{0.2\textwidth}
\emph{Innovation}
\end{minipage}

\begin{minipage}[c]{0.2\textwidth}
    \includegraphics[width=\textwidth]{Images/Social_sustainability/white_box.png}
\end{minipage}
\fbox{
\begin{minipage}[c]{0.5\textwidth}
9.6 Significantly increase access to information and communication technologies and strive to provide universal and low-cost access to the Internet in least developed countries by 2020
\end{minipage}
}
\begin{minipage}[c]{0.2\textwidth}
\emph{Digital transformation}
\end{minipage}

\begin{minipage}[c]{0.2\textwidth}
    \includegraphics[width=\textwidth]{Images/Social_sustainability/10_ineq.png}
\end{minipage}
\fbox{
\begin{minipage}[c]{0.5\textwidth}
10.3 Ensure equal opportunities for all and reduce result inequalities, including through the elimination of discriminatory laws, policies and practices, and the promotion of adequate laws, policies and actions in this regard

10.4 Adopt policies, in particular fiscal, wage and social protection policies, and progressively achieve greater equality
\end{minipage}
}
\begin{minipage}[c]{0.2\textwidth}
\emph{Evaluation, remuneration and incentive system}
\end{minipage}

\newpage
\begin{minipage}[c]{0.2\textwidth}
    \includegraphics[width=\textwidth]{Images/Social_sustainability/11_cities.png}
\end{minipage}
\fbox{
\begin{minipage}[c]{0.5\textwidth}
11.2 By 2030, provide access to safe, sustainable, and affordable transport systems for all, improve road safety, in particular by expanding public transport, with particular attention to the needs of those in vulnerable situations, women, children, people with disabilities and the elderly
\end{minipage}
}
\begin{minipage}[c]{0.2\textwidth}
\emph{Accessibility of service}
\end{minipage}

\begin{minipage}[c]{0.2\textwidth}
    \includegraphics[width=\textwidth]{Images/Social_sustainability/white_box.png}
\end{minipage}
\fbox{
\begin{minipage}[c]{0.5\textwidth}
11.3 By 2030, increase inclusive and sustainable urbanization and the capacity for participatory and integrated planning and management of human settlement in all countries
\end{minipage}
}
\begin{minipage}[c]{0.2\textwidth}
\emph{Community}
\end{minipage}

\begin{minipage}[c]{0.2\textwidth}
    \includegraphics[width=\textwidth]{Images/Social_sustainability/white_box.png}
\end{minipage}
\fbox{
\begin{minipage}[c]{0.5\textwidth}
11.6 By 2030, reduce the negative per capita environmental impact of cities, in particular with regard to air quality and waste management
\end{minipage}
}
\begin{minipage}[c]{0.2\textwidth}
\emph{Emissions}
\end{minipage}

\begin{minipage}[c]{0.2\textwidth}
    \includegraphics[width=\textwidth]{Images/Social_sustainability/12_consumption.png}
\end{minipage}
\fbox{
\begin{minipage}[c]{0.5\textwidth}
12.2 By 2030, achieve sustainable management and efficient use of natural resources
\end{minipage}
}
\begin{minipage}[c]{0.2\textwidth}
\emph{Consumption of energy and natural resources, waste management}
\end{minipage}

\begin{minipage}[c]{0.2\textwidth}
    \includegraphics[width=\textwidth]{Images/Social_sustainability/white_box.png}
\end{minipage}
\fbox{
\begin{minipage}[c]{0.5\textwidth}
12.6 Encourage businesses, especially large and transnational companies, to adopt sustainable practices and integrate sustainability information into their reporting
\end{minipage}
}
\begin{minipage}[c]{0.2\textwidth}
\emph{Governance of sustainability and supply chain}
\end{minipage}

\begin{minipage}[c]{0.2\textwidth}
    \includegraphics[width=\textwidth]{Images/Social_sustainability/13_climate.png}
\end{minipage}
\fbox{
\begin{minipage}[c]{0.5\textwidth}
13.2 Integrate measures to combat climate change into national policies, strategies and plans
\end{minipage}
}
\begin{minipage}[c]{0.2\textwidth}
\emph{Emission of GHG}
\end{minipage}

\begin{minipage}[c]{0.2\textwidth}
    \includegraphics[width=\textwidth]{Images/Social_sustainability/16_peace.png}
\end{minipage}
\fbox{
\begin{minipage}[c]{0.5\textwidth}
16.5 Substantially reduce corruption and all its forms

16.4 By 2030, significantly reduce illicit financing and arms trafficking, enhance the recovery and return of stolen assets and fight all forms of organized crime
\end{minipage}
}
\begin{minipage}[c]{0.2\textwidth}
\emph{Community}
\end{minipage}

\begin{minipage}[c]{0.2\textwidth}
    \includegraphics[width=\textwidth]{Images/Social_sustainability/white_box.png}
\end{minipage}
\fbox{
\begin{minipage}[c]{0.5\textwidth}
16.6 Develop effective, accountable and transparent institutions at all levels
\end{minipage}
}
\begin{minipage}[c]{0.2\textwidth}
\emph{Production and distribution of economic value}
\end{minipage}

\begin{minipage}[c]{0.2\textwidth}
    \includegraphics[width=\textwidth]{Images/Social_sustainability/white_box.png}
\end{minipage}
\fbox{
\begin{minipage}[c]{0.5\textwidth}
16.7 Ensure responsive, inclusive, participatory and representative decision making at all levels
\end{minipage}
}
\begin{minipage}[c]{0.2\textwidth}
\emph{Customer satisfaction and community attention}
\end{minipage}

\begin{minipage}[c]{0.2\textwidth}
    \includegraphics[width=\textwidth]{Images/Social_sustainability/white_box.png}
\end{minipage}
\fbox{
\begin{minipage}[c]{0.5\textwidth}
16.8 Promote and enforce non-discriminatory laws and policies for sustainable development
\end{minipage}
}
\begin{minipage}[c]{0.2\textwidth}
\emph{Ethics}
\end{minipage}

\newpage

\begin{minipage}[c]{0.2\textwidth}
    \includegraphics[width=\textwidth]{Images/Social_sustainability/17_partnerships.png}
\end{minipage}
\fbox{
\begin{minipage}[c]{0.5\textwidth}
17.13 Enhance global macro-economic stability, including through policy coordination and coherence

17.14 Improve policy coherence for sustainable development
\end{minipage}
}
\begin{minipage}[c]{0.2\textwidth}
\emph{Community attention}
\end{minipage}

\begin{minipage}[c]{0.2\textwidth}
    \includegraphics[width=\textwidth]{Images/Social_sustainability/white_box.png}
\end{minipage}
\fbox{
\begin{minipage}[c]{0.5\textwidth}
17.19 By 2030, build, on the basis of existing initiatives, systems for measuring progress towards sustainable development that are complementary to measuring GDP and support the creation of statistical capacity in developing countries

17.17 Encourage and promote effective partnerships in the public sector, between public and private and in civil society based on the experience of partnerships and their ability to find resources
\end{minipage}
}
\begin{minipage}[c]{0.2\textwidth}
\emph{Governance of sustainability}
\end{minipage}


\newpage

In order to better analyze these new aspects, the EU directive 2014/95\cite{directive201495eu} establishes that public interest companies, that meet a number of requirements, are required to provide a non-financial report. This annual sustainability report makes it possible to communicate the performance and sustainability impacts of a company. It consists of measuring, communicating and assuming responsibility towards stakeholders in relation to the performance of the organization with respect to the goal of sustainable development. The topic addressed in this report are:

\begin{itemize}
    \item Environmental topic
    \item Social and diversity equality topic
    \item Respect for human rights
    \item Anti-corruption and extortion
\end{itemize}

Given the importance of these new topics, the next paragraphs will be dedicated to identifying new indicators related to these new aspects. In particular, the \ref{sec:newtech} identifies the new technologies already on the market and other new technologies that are still in the research phase, but which will become more relevant in the future. Subsequently it is shown how new technologies and new data collected can lead to the definition of new indicators for green sustainability, social sustainability and maintenance.

Before identifying these new KPIs, it is good to remember the \emph{characteristics of a good indicator}.

First, indicators are measured to evaluate progress and they can be defined in terms of goals, objectives, targets and thresholds. For example, a planning process may involve establishing traffic congestion indicators (defining how congestion will be measured), goals (a desire for fast and efficient vehicle travel), objectives (changes in roadway supply or travel activity that reduces congestion) and targets (specific, feasible changes in congestion impacts or travel behavior that should be achieved), and thresholds (levels beyond which additional actions will be taken to reduce congestion).

Indicators can reflect various levels of analysis. For example, indicators may reflect the decision-making process (\textit{quality of planning}), responses (\textit{travel patterns}), physical impacts (\textit{emission and crash rates}), human and environmental effects (\textit{injuries and deaths, and ecological damages}), and their economic impacts (\textit{costs of crash and environmental damages}).
\newpage
Another way to classify the indicators in the following:
\begin{itemize}
    \item \emph{Process} the types of policies and planning activities, such as whether the organization has a process for collecting and publishing performance data, and public involvement.
    \item \emph{Inputs} the resources that are invested activities, such as the level of funding spent on various activities or modes.
    \item \emph{Outputs} direct results, such as the miles of sidewalks, paths and roads, and the amount of public transit service provided.
    \item \emph{Outcomes} ultimate results, such as the number of miles traveled and mode share, average travel speeds, congestion and crowding, number of accidents and casualties, energy consumption, pollution emissions, and user satisfaction.
\end{itemize}
It is often best to use some of each type of performance indicators. 

Another aspect to consider is the use of qualitative or quantitative data. Quantitative data refers to easy-to-measure information, while qualitative data refers to other types of information and they can be quantified using letter or number ratings. 

Relative indicators are often used to evaluate many impacts, such as trends or comparisons with peers. Reference units (also called ratio indicators) are measurement units normalized to facilitate comparisons, such as per-year, per-capita, per-mile, per-trip, per-vehicle-year and per euro. The selection of reference units can affect how problems are defined and solutions prioritized. For example, measuring impacts such as emissions, crashes and costs per vehicle-mile ignores the effects of changes in vehicle mileage. Measuring these impacts per-capita does account for changes in vehicle travel.
\newpage
Summing up the \emph{principles to consider selecting a new KPI} are:
\begin{itemize}
    \item Relevance to the community's definition of sustainability, considering the suburban or urban area
    \item Understandability by the community at large, not only experts
    \item Acceptance and usage by the community
    \item Long-term view of the community
    \item Integration of different relevant topic
    \item Based on information that is reliable, accessible, timely and accurate
    \item Consideration of local impacts at the expense of global impacts
    \item Quantitative KPIs are easier to analyze, they are considered more objective and they tends to receive more weight in a planning process. Sustainability indicators therefore require quantifying impacts as much as possible.
    \item Trade-off between smaller set of indicators, using available data, and larger set can be more comprehensive but have excessive data collection and analysis costs. 
\end{itemize}

\newpage
\section{New Public Transport Technologies}
\label{sec:newtech}
In this paragraph, as previously stated, we are going to analyze the public transport future requirements that, in term of technologies, are becoming increasingly stringent. In order to be competitive on the bus market, the vehicles have to be equipped with a lot more technologies than nowadays. The technologies are needed since the world now is demanding more an more data and insights on every activity, not only from a generic and distant point of view, but from a direct, punctual and real-time view. Nowadays Public Transport in Italy, and buses in particular, are often equipped with close to nothing technologies that can improve the service; as we have seen in the previous chapters, Arriva systems used to collect the data in a quite old way: all the spreadsheets concerning the service, control management, balance sheets are clearly written by hand.

Public Transport technology innovation is not just the systems that can be installed onboard, such as the AVL or APC, but also all the world that revolve around them, from the service’s planning and management to the data gatherings and costumer feedbacks. All the innovative technology can be divided in two big groups: the systems that are installed directly on the buses, and the systems that operate remotely, off the buses.


\subsection{Technologies installed on bus}
\label{subsec:techonbus}

\begin{figure}[h!]
    \centering
    \includegraphics[width=0.4\textwidth]{Images/New Technologies/immagine intro Tech On.png}
    \caption{Technology Installed on Bus}
    \label{fig:onbus}
\end{figure}


The purpose of on – bus technology is mainly service monitoring and data gathering for demand analysis and offer adjustments, those type of system have become particularly relevant in the, very recent, Covid–19 period where it was particularly relevant analyze and report the mobility evolution to prospect near future scenarios and assess the effectiveness of adopted actions.


\paragraph{AVM and AVL}
Automated Vehicle Monitoring is a technology based on localization through GPS, monitoring and recording different topics related to moving vehicles, such as position, speed, diagnostic of mechanical components and so on. Automated Vehicle Location is more focused on the GPS position on the vehicle but if GPS signals are poor, AVL can use different technologies to determine actual location information, such as dead reckoning, which takes a previously determined position, and then incorporates estimations of speed, heading direction, and course over elapsed time. 

The buses, based on the GPS signal that the satellite sends, calculate their position a certain amount of seconds. This information, together with other data, is sent to the operator in the operation centre which operate in real-time the data and sends them to the costumer information management systems. Those data are also stored in order to make them available to the procuring entity for service reporting.

This type of technology allow the PTO to increase the security level and control in case of disruptive events, being able to handle and analyze big amount of data to assess a quick result or motivation.

Those technologies can be used to compute some interesting KPIs, tacking the status of the service, advances and delays, vehicle tracking, service adjustments in case of disruptive events. Also, from a graphical point of view the real-time situation can be displayed really well and it can be used widely in a dashboard.
\begin{figure}[H]
    \centering
    \includegraphics[width=0.7\textwidth]{Images/New Technologies/AVMeAVL.jpg}
    \caption{AVM and AVL\cite{avmavlimmage}}
    \label{fig:avmavl}
\end{figure}

\paragraph{Automatic People Counter}
People counters are electronic sensors that count people as they walk into an entrance. There are various methods for those systems to count people. Those can be classified into contact-type counters, sensors implemented system, and vision-based system using a camera. Contact-type counters are mechanical counters which need human contact such as turnstiles and mat-type foot switches that can obstruct the path. 

However, this type of counter only applicable for minimal people counting and it is not suitable for massive number of people, such as a high flow of people boarding or getting off the bus. Other technologies used in people counter are Pyroelectric Infrared (PIR) sensors, ultrasonic sound distance sensor and thermal sensor. PIR sensor counts people when they pass through the area of observation which is mounted with a pair of IR transceiver. Thermal sensors are only able to observe the crowd density without counting them accurately. 

\begin{figure}[h!]
    \centering
    \includegraphics[width=0.5\textwidth]{Images/New Technologies/APC.jpg}
    \caption{Automatic People Counter\cite{apcimage}}
    \label{fig:APC}
\end{figure}

People counting technologies can be used for many purposes: assessing that is impossible to have them on all the lines of the consortium, it is possible to use a data analysis on routes that do not have APC to build a model to estimate the level of service onboard on all routes. The load factor information is very useful to consult on a real-time dashboard, creating a trend of the number of people on board; in this way it is also possible to study the performance of the stops, as well as the lines, creating a set of KPIs that can function both as real-time  indicators and for decision-making at a future planning level.

\paragraph{Video–surveillance }
Video-surveillance is a technology that is extremely helpful for public transport management; cameras can be installed in three main positions: outdoor cameras, to monitor the surrounding scene and in particular the blind spots, indoor cameras, to monitor bus passengers, and driver cameras, to monitor the driver behaviour. All those images can be provided to the driver with clear images referring to these areas, which can then be displayed on special monitors to be installed on the driver's dashboard. 

With the new telecommunication technologies, like 4G and 5G, it can be performed a sort of edge computing through the help of video analysis with AI systems; these methods can lead to fast and responsive actions or adjustment that the drivers, or the employees in remote, can do. Artificial Intelligence algorithms can detect some risk factors for what concern the ride’s safety, such as tiredness, distraction, or irregular behaviour of the driver.

\begin{figure}[h!]
    \centering
    \includegraphics[width=0.6\textwidth]{Images/New Technologies/VIDEOSURV.PNG}
    \caption{Video–surveillance\cite{videosurvimage}}
    \label{fig:vs}
\end{figure}


\paragraph{Electronic ticketing systems}
The electronic ticketing system is not a proper technology installed just on buses, but the validation and control parts are managed largely on the bus. The system is composed by the validators and the On-Board Computer, which are installed directly on the buses, and by a central Control Center where all the data collection, management and distribution of incomes and emission or recharge of smart cards takes place.

\begin{figure}[h!]
    \centering
    \includegraphics[width=0.45\textwidth]{Images/New Technologies/ELECTRONIC TICKET.png}
    \caption{Electronic ticketing systems\cite{eticketimage}}
    \label{fig:ets}
\end{figure}

One of the trending topics on the whole ticketing system is the type of support used: the support is the connection between costumers and PTO, which means that more information the support can bring to the PTO the better it is. In recent times there has been a rapid evolution of the support type, starting from basic paper, passing through magnetic, smart-cards and lastly mobile tickets. With Near-Field Communication (NFC) technology, the fares are redeemed without contact and completely remove the need to handle any physical cash or fares, and even limit the interaction needed with the driver. An additional motivation is that it is been found that physical money can carry the virus (in particular Covid-19) for up to three days, and by bypassing the need to purchase paper tickets and validate them, mobile ticketing can help reduce the transmission of the virus between riders and staff. An incredible by-product of integrating this technology is also the impact it has on the environment, as physical tickets no longer need to be created.

Keeping the user in mind, public transit technology is moving toward creating a more personalized ridership experience. A shift in this space is Account-Based Automatic Fare Collection. Through this system, all travel history, documents, and account information for riders is gathered in a customized dashboard.

With this technology, passengers can identify and participate with every available public service that is enabled for Account-based fare collection, making it far easier to use different modes of transportation, and save specific routes or journeys for use later. This also removes the hassle of having to manage multiple user accounts, payment methods, and receipt/billing information.

\subsection{Technologies installed off bus}
\label{subsec:techoffbus}
In addition to technologies directly installed on buses, there are some systems that play a center role in the world of public transport innovation, even not being directly on the public transport vehicle. Those type of technologies are aimed more at the network part of the service: we are talking about all the secondary, or subsidiary, technology that are vastly growing in the world, that together form the infamous Internet of Things. 

The general concept is that public transportation is affected by countless variables and not only by the ones directly connected to it, such as the costumers, the bus health, the drivers and so on, for example: the weather is clearly a source of data in correlation, but more importantly in causation, with the transportation system, if a day there is a particularly heavy rain, people tend to use more the private vehicle, using less the bus or in general the public transportation, this is due to the fact that often using PT means waiting outside under the rain. This situation have a great impact on the PTO systems, in fact bus tend to be less populated, so all the KPIs related to the number of costumers onboard are affected, as well as the travel time since, as said before, there are more private vehicles on the road than in a normal day; travel time therefore will be higher than usual and more delays, all of this affect the KPIs related to the commercial speed, travel time, delays per day and so on \dots

This big example is made to understand that the more data of different types are gathered daily, the more precise the analysis that a PTO can assess. In this new era of technology, especially now with the faster growing 5G, it is very important to take advantage of them, exploiting all of their potentialities.

The potential of digital applications and solutions for public transit is endless. Sensing devices and IoT technology are becoming more robust and low-cost, providing useful data for municipalities, fleet managers and even riders. For example, in Sweden a recent study used wireless sensors on city buses to monitor real-time air pollution. The sensors gave more accurate data on highly polluted urban areas, which led to better-informed city planning decisions and air quality improvement efforts. 

Telematics technology is also becoming a way to provide transportation fleet managers with real-time data to help identify and address things like temperature control, fuel levels, optimal driving routes, operating efficiencies and more. Predictive analytics allows for quick maintenance and better energy management, which is critical to addressing environmental impact.

Other important aspects on this topic are the actions that the PTA can take directly, for example, with the spread of smart cities, the necessary information and communication infrastructure lends itself well to intelligent traffic management systems. These systems allow centralised traffic control, enabling transit authorities to intelligently manage traffic lights, cameras, emergency routes and public transport routes.

Urban transit systems are also looking into how to incorporate rapid transit into existing systems. While this term most often refers to rapid transit trains or hyperloop systems, this can also refer to rapid transit buses, which have dedicated traffic lanes and priority at intersections; something that smart traffic management could help manage.


\subsection{New KPIs possibilities}
\label{subsec:newpossibilities}
As we have understood in the previous paragraphs, frequency in data–gathering is a key factor. Looking at the data provided to us is almost surprising that a company like Arriva, and all the consortium in general, has such a lack of detail in the available data. The dashboard built in the previous chapter has a potential, clearly limited by the frequency and the quality of what’s inside.

But why a higher frequency data – gathering is the base answer to most of the problems? Gathering more data more frequently means that it becomes much easier to understand trends, seasonalities, or particular micro – behavior that a monthly, or in some cases even yearly frequency cannot pick. In particular, algorithms like regression, classification, clustering, anomaly detection, and many others, work so much better with large quantities of data, and therefore understanding what they have to tell us is far easier.

Another strong motivation for why is essential to have data gathered more frequently is the possibility of expansion of the analysis that can be done: looking at a more detailed time span is useful to build a model on different types of time centered analysis, such as workdays vs. weekdays, peak hour vs. off-peak hours. The possibilities of Arriva, as now, are to compute KPIs based on the company trend of the month or year, which can hide an enormous quantity of information; having daily, hourly, or even more precise data, allows the analysts, and us, to build a dashboard with better KPIs.

\paragraph{Dynamic KPIs}
Other than more precise KPIs, gathering all those types of data can mean also being able to compute different \textit{types} of KPIs: we are used to looking at dashboards with basic cardinal numbers, which by definition are static and provide agglomerative information on the subject, such as a mean, a median, maximum or minimum and so on. 

Introducing the use of \textit{Dynamic KPIs} opens up a new dimension in the data – plane in which the analyst can look and assess more types of conclusions. The Dynamic KPI is a particular way to view a result where the user can look directly at the trend, in real-time of the data, but let’s make an example.

If we want to analyze the number of passengers onboard on a specific ride, until now the KPIs were all built as a mean number of the occupation onboard, representing the average number of passengers on that ride. But this information, which is still useful, don’t get us wrong, is not able to tell the users how the flow of passengers distributes among the stops of the ride. Using a dynamic KPIs that shows both the people loaded and unloaded at every stop, we can see in the data more information, such as a particular section in the run where there are very few people onboard, and if it is behavior repeated in time, this information is useful to assess some conclusion, such as remove some stops, add new ones and many more.

\begin{figure}[t]
    \centering
    \includegraphics[width=1\textwidth]{Images/New Technologies/dynamic KPIs.png}
    \caption{Dynamic KPIs Example}
    \label{fig:dynKPI}
\end{figure}

As it is possible to see from the image  \ref{fig:dynKPI}, both are telling the same information, but in two different ways: the number on the right tells us just the mean number of passengers for that particular ride, although looking at the graph on the left we can detect much more information: the central stops of the ride (7, 8 and 9) are very little used, and also the number of passengers on board at that moment are almost zero. Here the PTO can make multiple decisions, such as dividing the line into two pieces to maximize the effectiveness of both of them. A deeper analysis can be done to understand why those stops are rarely used, maybe they can be re-positioned to meet higher demand, changing the line route by a little.




% Please add the following required packages to your document preamble:
% \usepackage{lscape}
\newpage
\thispagestyle{empty}
\begin{landscape}
\begin{table}[]
\centering
\begin{tabular}{|l|l|l|l|}
\hline
\rowcolor{bluepoli!40}
\multicolumn{1}{|c|}{\textbf{KPI denomination}}                                       & \multicolumn{1}{c|}{\textbf{Technologies Used}} & \multicolumn{1}{c|}{\textbf{Unit of Measure}} & \multicolumn{1}{c|}{\textbf{Explanation}}                                                                                                                                                          \\ \hline
\begin{tabular}[c]{@{}l@{}}Daily tickets sold   \\ over passenger moved\end{tabular}  & Iot, Electronic Ticketing,   APC                & €/\# of people                                & \begin{tabular}[c]{@{}l@{}}Understand how the amount \\ of ticket sold affect \\ the overall passenger,\\  helping to size the service   \\ not only on historical \\ data and passes\end{tabular} \\ \hline
\begin{tabular}[c]{@{}l@{}}Distribution of   \\ ticket vs. pass per ride\end{tabular} & Electronic   Ticketing, APC                     & \%                                            & \begin{tabular}[c]{@{}l@{}}Understand which   \\ type of costumer takes \\ certain rides allow to\\  assess decision for a ride\end{tabular}                                                       \\ \hline
\begin{tabular}[c]{@{}l@{}}Annual Fuel cost  \\  over km done\end{tabular}            & AVM, AVL                                        & €/km                                          & \begin{tabular}[c]{@{}l@{}}Helps understand the \\   driving behavior of the drivers \\ and the cost percentage \\ that affect the   balance sheets\end{tabular}                                   \\ \hline
\end{tabular}
\caption{List of new Economical and Financial KPIs}
\label{tab:economic}
\end{table}
\end{landscape}
\newpage
% Please add the following required packages to your document preamble:
% \usepackage{lscape}
\newpage
\thispagestyle{empty}
\begin{landscape}
\begin{table}
\centering
\begin{tabular}{|l|l|l|l|}
\hline
\rowcolor{bluepoli!40}
\multicolumn{1}{|c|}{\textbf{KPI denomination}}                                                    & \multicolumn{1}{c|}{\textbf{\begin{tabular}[c]{@{}c@{}}Technologies \\ Used\end{tabular}}} & \multicolumn{1}{c|}{\textbf{\begin{tabular}[c]{@{}c@{}}Unit of\\  Measure\end{tabular}}} & \multicolumn{1}{c|}{\textbf{Explanation}}                                                                                                                                                        \\ \hline
\begin{tabular}[c]{@{}l@{}}Mean delay by time slot \\ (peak vs. off-peak) and ride\end{tabular}    & AVL, AVM                                                                                   & min                                                                                      & \begin{tabular}[c]{@{}l@{}}Understanding if a  \\delay is mainly caused \\ by the departing hour of the ride, \\ can help to modify   the schedule according to it\end{tabular}                    \\ \hline
\begin{tabular}[c]{@{}l@{}}Visualization on where \\ the ride generates the delay\end{tabular}     & AVL, AVM                                                                                   & map                                                                                      & \begin{tabular}[c]{@{}l@{}}Deepening the   knowledge of the delay \\ can mean even \\better improvements \\ going to make specific   changes \\ in some section of the lines\end{tabular}          \\ \hline
Lost costumers per   suppressed ride                                                               & APC, AVM                                                                                   & \#                                                                                       & \begin{tabular}[c]{@{}l@{}}Knowing how many   \\people are affected \\ by a suppressed ride\\ can help to assess \\ the importance of   that ride and an overall \\ level of discomfort\end{tabular} \\ \hline
Delay amount over mm   of rain                                                                     & AVM, IoT                                                                                   & min/mm                                                                                   & \begin{tabular}[c]{@{}l@{}}Knowing the relation  \\ between delays and \\ weather conditions\\ is important to correctly \\ dimension the   problem\end{tabular}                                     \\ \hline
\begin{tabular}[c]{@{}l@{}}Load Factor per run   \\ per time slot (peak vs. off-peak)\end{tabular} & APC, AVM                                                                                   & \%                                                                                       & \begin{tabular}[c]{@{}l@{}}Knowing the   percentage of load\\  for a particular run can help to allocate \\ better the resources\end{tabular}                                                    \\ \hline
Load Factor vs.   Weather Condition                                                                & APC, IoT                                                                                   & \%                                                                                       & \begin{tabular}[c]{@{}l@{}}Also in this case, knowing \\   the behavior in\\ different weather condition \\ can help to allocate better the resources\end{tabular}                                 \\ \hline
\end{tabular}
\caption{List of new Service Quality KPIs}
\label{tab:servicequality}
\end{table}
\end{landscape}
% Please add the following required packages to your document preamble:
% \usepackage[normalem]{ulem}
% \useunder{\uline}{\ul}{}
% \usepackage{lscape}
\newpage
\thispagestyle{empty}
\begin{landscape}
\begin{table}[]
\centering
\begin{tabular}{|l|l|l|l|}
\hline
\rowcolor{bluepoli!40}
\multicolumn{1}{|c|}{\textbf{KPI denomination}}                                                        & \multicolumn{1}{c|}{\textbf{Technologies Used}} & \multicolumn{1}{c|}{\textbf{Unit of Measure}} & \multicolumn{1}{c|}{\textbf{Explanation}}                                                                                                                                                                                                                             \\ \hline
\begin{tabular}[c]{@{}l@{}}Percentage of low environmental   \\ impact bus on the road\end{tabular}    & AVM                                             & \%                                            & \begin{tabular}[c]{@{}l@{}}Knowing how many green   bus \\ are circulating on the road can be \\ an interesting KPI to show at the PTA\end{tabular}                                                                                                                   \\ \hline
Percentage of km   made by hybrid/EV bus                                                               & AVM, AVL                                        & \%                                            & \begin{tabular}[c]{@{}l@{}}Same as before, knowing  how many kms \\ have been done in a healthy way is useful\end{tabular}                                                                                                                                            \\ \hline
Tons of $CO_2$ emitted   by bus or by line                                                             & AVM                                             & ton                                           & \begin{tabular}[c]{@{}l@{}}Understanding which   model of bus on which line\\  is the most emissive is useful to assess \\ how to renovate   the fleet\end{tabular}                                                                                                   \\ \hline
\begin{tabular}[c]{@{}l@{}}Tons of $CO_2$   \\ emitted in the city center vs. rural areas\end{tabular} & AVM, AVL                                        & ton                                           & \begin{tabular}[c]{@{}l@{}}Knowing the spatial   distribution of tonnage \\ emission can be useful to re-allocate different bus  \\  models on the lines\end{tabular}                                                                                                 \\ \hline
\begin{tabular}[c]{@{}l@{}}Mean bus consumption   \\ per run, per line, per bus type\end{tabular}      & AVL, AVM                                        & $L_{fuel}$/(run,   line, bus)                 & \begin{tabular}[c]{@{}l@{}}Awareness on bus   consumption by run is useful \\ to re-allocate and renovate the rolling stock\end{tabular}                                                                                                                              \\ \hline
Energy consumption curve   for EV bus                                                                  & AVM, AVL                                        & kWh/(time, km)                                & \begin{tabular}[c]{@{}l@{}}Having Electric Vehicle can be difficult to manage \\ in terms of recharging point and time allocation,  \\  understanding where the bus consume most \\ of its energy is handy to optimally   place \\ the recharging points\end{tabular} \\ \hline
\end{tabular}
\caption{List of new KPIs on Rolling Stock}
\label{tab:rollingstock}
\end{table}
\end{landscape}
% Please add the following required packages to your document preamble:
% \usepackage{lscape}
\newpage
\thispagestyle{empty}
% Please add the following required packages to your document preamble:
% \usepackage{lscape}
\begin{landscape}
\begin{table}
\centering
\begin{tabular}{|l|l|l|l|}
\hline
\rowcolor{bluepoli!40}
\multicolumn{1}{|c|}{\textbf{KPI denomination}}                                                       & \multicolumn{1}{c|}{\textbf{Technologies Used}} & \multicolumn{1}{c|}{\textbf{Unit of Measure}} & \multicolumn{1}{c|}{\textbf{Explanation}}                                                                                                                                                             \\ \hline
\begin{tabular}[c]{@{}l@{}}Commercial speed \\ deviation   from mean\end{tabular}                     & AVL, AVM, historical   data                     & $\pm$ km/h                                    & \begin{tabular}[c]{@{}l@{}}Comparing real-time   \\ commercial speed with the mean \\ of historical data help understand, \\ together   with delays information, \\ where is the problem\end{tabular} \\ \hline
Amount of km per day                                                                                  & AVM                                             & km                                            & Amount of kilometers   per day                                                                                                                                                                        \\ \hline
Planned vs. Effective   kms                                                                           & AVM                                             & \%                                            & \begin{tabular}[c]{@{}l@{}}Comparison between   \\ planned km on schedule \\ and km effectively done in a day\end{tabular}                                                                            \\ \hline
\begin{tabular}[c]{@{}l@{}}Mean time per stop   \\ (peak vs. off-peak)\end{tabular}                   & AVL                                             & sec                                           & \begin{tabular}[c]{@{}l@{}}Amount of time taken   at each stop to \\ loading and unloading the passengers\end{tabular}                                                                                \\ \hline
\begin{tabular}[c]{@{}l@{}}Stop time in   relation to number \\ of people moved per stop\end{tabular} & AVL, APC                                        & \# of people / sec                            & \begin{tabular}[c]{@{}l@{}}As the one above but    comparing it \\ to the number of people \\ that   actually moved\end{tabular}                                                                      \\ \hline
\begin{tabular}[c]{@{}l@{}}Commercial speed vs.\\ millimeters of rain/snow\end{tabular}               & AVL, IoT                                        & f(Km/h, mm)                                   & \begin{tabular}[c]{@{}l@{}}Comparison between mean  \\  commercial speed \\ of the ride and the amount \\ of rain fall in that period of   time\end{tabular}                                          \\ \hline
\begin{tabular}[c]{@{}l@{}}Commercial speed vs. \\ Fuel Consumption or $CO_2$ produced\end{tabular}   & AVL, AVM                                        & (km/h)/kWh, (km/h)/$tons_{CO\_2}$           & \begin{tabular}[c]{@{}l@{}}Helps understand the   driving behavior \\ of the drivers among with the points \\ where to optimize it\end{tabular}                                                       \\ \hline
\end{tabular}
\caption{List of new Operational KPIs}
\label{tab:operation_indication}
\end{table}
\end{landscape}

\subsection{Mock Up Dashboard}
The aim of this section is to built and visualize some mock-up dashboard pages, starting from the KPIs listed tables \ref{tab:economic}, \ref{tab:servicequality}, \ref{tab:rollingstock}, \ref{tab:operation_indication}.

We would like to point out that all pages are a mock-up also for what concerns the data present in he graphs, which may appear not close to the reality and made on purpose to talk about their benefits.

\newpage

\newpage
\begin{landscape}
\thispagestyle{empty}
\includepdf[angle=90, pagecommand={\null\enlargethispage{3\baselineskip}\vfill\captionof{figure}{Economical page}\label{fig:ep}}]{mockup/economical.pdf}
\end{landscape}


\subsubsection{Economical page}
In this page we have created a visualization, in addition to the traditional one, able to tell a lot more information about the daily revenues of the company, for example, in the graph upright (\emph{Daily single tickets sold per passenger moved}) is possible to classify the dots (which each one represents a day) in four main areas:

\begin{itemize}
    \item \textcolor{blue}{\textbf{Blue Area}} is the normality, where average revenues are collected and the passengers moved are around a standard mean, depending, for example, from the day of the week of the time of the year.
    \item \textcolor{red}{\textbf{Red Area}} instead can represent the moment during the year where most people buy their monthly or yearly pass, which cause an abnormal amount of revenues, but still the passengers carried are quite constant.
    \item \textcolor{green}{\textbf{Green Area}} can be special day of the year, when maybe ticket office are closed (so less revenues from those), but a lot of people travel with public transport.
    \item \textcolor{cyan}{\textbf{Light Blue Area}} can represent some particular days when events happen, if Atalanta BC plays an important home game, a lot of supporters can choose to reach the stadium using public service, causing an substantial increase in the amount both in revenues and in passengers moved.
\end{itemize}

The graph in the lower-left part of the page shows the distribution of users of a particular line among its ride in the day. By visualizing how many people using different type of tickets have taken the bus, its possible to assess some actions regarding the advertising or the better sizing of the bus. So actions that can work on different levels: operational and marketing-wise.

\newpage
\begin{landscape}
\thispagestyle{empty}
\includepdf[angle=90, pagecommand={\null\enlargethispage{3\baselineskip}\vfill\captionof{figure}{Service Quality page}\label{fig:sqp}}]{mockup/service_quality.pdf}
\end{landscape}

\subsubsection{Service Quality page}
Understanding the quality of service is a key factor both for the operational and economical side of the company, but also for the costumer satisfaction. Assessing its behavior daily can be a big help in order to understand all the minor actions in which the company can work on.

This page mainly focus on the \emph{delay} and its variablily taking into account:
\begin{itemize}
    \item Time of the Day: Peak and Off Peak Hours
    \item Weather Conditions: millimeters of rain (or snow)
    \item Location of the delay accumulation
\end{itemize}

Those three information together can help to visualize how and where the rides \emph{accumulate the delay}: knowing if there is a relation between minutes lost and millimeters of rain can, for example, help to a better planning on a day to day basis. Instead, for example, knowing \emph{where is located} the delay accumulation is useful to assess some major changes on the line path or work together with the PTA to find a more powerful solution.

\newpage
\begin{landscape}
\thispagestyle{empty}
\includepdf[angle=90, pagecommand={\null\enlargethispage{3\baselineskip}\vfill\captionof{figure}{Rolling Stock and Operation page}\label{fig:sqp}}]{mockup/rolling_stock.pdf}
\end{landscape}

\subsubsection{Rolling Stock and Operation page}
This last page is more bus-centered, both looking at how the company is managing the fleet to be put on the road and assessing how well the fleet is performing looking at the Commercial Speed and the Energy Consumption curve.

The amount of green buses that a company have in their fleet is, and will be more and more, an important issue, we wanted to emphasize that by creating a visualization where is possible to know the percentage of Green Buses on road and with a visualization of the rolling stock composition.







\newpage
\section{Social Sustainability}
Social sustainability is a frequently disregarded part of sustainability, as supportable advancement discussions frequently centre around the natural or monetary parts of supportability. In fact, all three dimensions of sustainability must be addressed to attain the most sustainable outcome possible.

Social sustainability is a process for creating sustainable and successful places that promote well-being by understanding the needs of people in the places where they live and work. Social sustainability combines the design of the physical environment with the design of the social world: infrastructure to support social and cultural life, social services, systems for engaging citizens and spaces for people and places to evolve. 

A Public Transport company like Arriva is in constant contact with its customers, whose well-being is a key aspect of the company's success. it is therefore important to assess all the aspects of social sustainability, both on the customer side but also on the internal side of the company, concerning the employees.

Having said that, we will analyze three main aspects on this topic:

\begin{itemize}
    \item Workers’ Sustainability
    \item Quality of Service
    \item Cleanliness
\end{itemize}

\subsection{Workers’ Sustainability}
Employee sustainability is the current and future ability of workers to remain in the workforce and is determined by a healthy organizational culture that supports and values employees. A sustainable employee culture keeps employees engaged to the level needed to perform their jobs capably.

Organizations that are concerned with employee sustainability recognize the need to create environments where employees remain engaged, perform at a high level, and experience job satisfaction, job commitment, and cultural buy-in throughout the duration of their employment. 

As the workforce ages, more emphasis is being placed on the importance of sustainable employment, since workers of all age groups, including those just entering the workforce, are making employee sustainability a priority in their job search.

More and more job seekers are looking to join companies that are concerned not only with environmental sustainability but are dedicated to building a culture of employee sustainability within their organizations as well.

As introduced before in \ref{sec:Intro_Innvovative}, the Sustainable Development Goals are set reach many goals at humanity level, one of which is Social Sustainability; below are listed some of the SGDs that take into account specifically the workers’ sustainability.

\begin{figure}[h!]
    \centering
    \includegraphics[width=1\textwidth]{Images/Social_sustainability/SGD workers sustainability.PNG}
    \caption{SDGs related to Workers' Sustaibability}
    \label{fig:SDGworksus}
\end{figure}

\begin{description}
   \item[n°1 - No Poverty] According to the 1.3 SDG which aim at being able to defeat poverty, the salaries must be adjusted to the job position according to the national collective labour agreement. To monitor this aspect is useful to consider the following KPI:
   \begin{itemize}
       \item Average salary level in proportion to the national collective labour agreement
   \end{itemize}
   \item[n°4 - Quality Education ]The 4.4 SDG emphasises the importance of training education in workplace in order to have always better quality of service and to increase safety of workers and clients.
    \begin{itemize}
       \item Training hours per capita for traveling and non-traveling personnel
       \item \% Employees who received an evaluation during the year
   \end{itemize}
   \item[n°5 - Gender Equality ] In particular 5.5 SGD, prohibit any forms of discrimination and gender disparities in company employees (5.5 SDG). Some reference indicators to consider measuring the gender equity are the following:
   \begin{itemize}
       \item \% women in management area: more importance on the women full and effective participation and equal leadership opportunities at all levels of decision-making in political, economic and public life
       \item \% women with a permanent contract
       \item Ratio between training hour for woman and men
   \end{itemize}
   \item[n°8 - Decent Work and Economic Growth ] The 8 SDG remark the importance of an equal and fair salary for all workers, including young people and people with disabilities and the importance to reduce the percentage of unemployed young people who do not follow a course of study or who do not follow training courses. Going more in detail, the 8.2 SDG stresses the importance to achieve higher levels of economic productivity through diversification, technological updating and innovation, including through a focus on high value-added sectors and labour-intensive sectors. 
   \begin{itemize}
       \item \% employees with permanent contract
       \item \% employees under 35 
       \item \% turnover rate
       \item Frequency of injuries of workers
       \item Index of severity of injuries
       \item Number of hours of training on safety and health
       \item \% employees registered with trade unions
       \item Number of hours of trade union assemblies
       \item Complaints related to work practices
   \end{itemize}
   \item [n°9 - Industry, Innovation and Infrastructure ] The importance of accessing to information and communication technologies is remarked by the 9.6 SDG. In a transport company is essential to master this goal since it could provide a great amount of benefits in terms of real-time communication and feedback from the drivers on the road.
   \begin{itemize}
       \item \% Employees with company phone with application used by the company
   \end{itemize}
\end{description}

\subsection{Quality of Service}
For a long time, the performance evaluation of Public Transport (PT) has been carried out from the service managers’ perspective, based on the cost efficiency and cost effectiveness of PT services and operations. However, in the last few decades, Service Quality (SQ) has become a major area of attention for practitioners, managers and researchers, who have focused on the passengers' perspective.

The perception of Service Quality is the result of a comparison of consumer expectations with actual service performance perception or with ideal performance, depending on which type of approach we would look at. The existence of different methodologies could be justified by the complexity of the Service Quality concept, the number of attributes used to evaluate it, or the imprecision and subjectivity of the data used to analyse it, typically based on customer satisfaction surveys (CSS). 

We have considered two approaches at Service Quality in this report, one using the above cited SDGs that talk about this aspect and the other taking into account “Carta della Mobilità” of Arriva, produced every year, which is the document that regulates the relationship between PTO and the users.

\begin{figure}[h!]
    \centering
    \includegraphics[width=0.6\textwidth]{Images/Social_sustainability/SDG service quality.PNG}
    \caption{SDGs related to Quality of Service}
    \label{fig:SDGqualiserv}
\end{figure}

\begin{description}
   \item[n°3 - Good Health and Well-Being] The 3.6 SDG has the goal to halve the number of deaths and injuries from road traffic accidents worldwide by 2020. To monitor this aspect the following KPIs can be considered: 
   \begin{itemize}
       \item Number of injuries and deaths caused by road accidents
       \item Number of road accidents
   \end{itemize}
   \item[n°5 - Gender Equality]In SDG 5.1 the main focus is to end all type of discrimination, and for our purposes we have thought of some KPIs that can be useful to monitor for a PTO on this aspect: 
    \begin{itemize}
       \item \% female passengers: this information can be collected easily if the passenger register on the website or on the application of the company 
       \item Number of assaults and violence to women on buses
   \end{itemize}
   \item[n°16 - Peace, Justice and Strong Institution ]The has the goal by 2030, to reduce illicit financing and arms trafficking, enhance the recovery and return of stolen assets and fight all forms of organized crime, reduce corruption and all its forms (16.4), Develop effective, accountable and transparent institutions at all levels (16.6). 
   
   To monitor fare evasion, corruption and arms trafficking:
    \begin{itemize}
       \item Number of passengers checked
       \item Number of clients connected on social network
       \item Number of accesses to the site
       \item Courtesy of drivers, this information is already collected with the customer satisfaction survey
       \item Number of info point in the area 
       \item Number of point of sales of tickets in the area
   \end{itemize}
\end{description}

Then, as reported in the document \textit{Carta della Mobilità}, the quality of the service can be perceived through a series of fundamental factors that characterize the quality of each aspect of the trip (e.g. travel safety, regularity of the service, cleanliness and hygienic conditions of the vehicles, etc.) and, within each of these, by specific quality indicators (for example for travel safety: number of accidents, age of vehicles) which represent the performance levels of the service provided.

Each factor and quality indicator are associated with a value (which expresses the level of quality of the service actually provided) and a target set each year by the Company providing the service.
The data on customer satisfaction are collected, in accordance with the provisions of the service contracts, with surveys carried out every six months by an external market research company through response surveys. 

Going more in detail, the surveys analyse the following aspects with their respective KPIs:

\begin{itemize}
    \item {Safety of the trip}
\begin{itemize}
    \item Accidents of the vehicles
    \item Passive accidents
    \item Age of vehicles
    \item \textit{\textbf{Total perceived safety of the trip}}
\end{itemize}
    \item Personal and property safety
    \begin{itemize}
        \item	Complaints (theft and harassment)
        \item \textit{\textbf{Total perceived personal safety}}
    \end{itemize}
    \item Regularity and punctuality of the service
    \begin{itemize}
        \item Regularity of the service
        \item Frequency of rides
        \item Commercial speed
        \item Punctuality in rush hours
        \item Punctuality in non-rush hours
        \item \textit{\textbf{Total perceived regularity of the service}}
    \end{itemize}
    \item	Comfort and cleanliness of the buses
    \begin{itemize}
        \item Crowding in rush hours 
        \item Crowding in non-rush hours 
        \item Air conditioning
        \item Low-floor bus
        \item Additional services (mobile platform, wheelchair anchoring)
        \item \textit{\textbf{Total perceived comfort}}
        \item Ordinary cleaning
        \item Extraordinary cleaning
        \item Cleaning of bus stations
        \item \textit{\textbf{Total perceived cleanliness}}
    \end{itemize}
    \item Information and service to users
    \begin{itemize}
        \item Timeliness
        \item Internal visual devices
        \item Timetable at bus stops
        \item Point of sales of tickets
        \item Feedback to complaints
        \item \textit{\textbf{Total perceived information and services}}
    \end{itemize}
    \item Relational aspects
        \begin{itemize}
            \item  \textit{\textbf{Total perceived relational aspects}}
        \end{itemize}
    \item Attention to environment
        \begin{itemize}
            \item Electric or hybrid vehicles 
            \item Use of eco-fuels
            \item Vehicles Euro 3-4
            \item Vehicles Euro 4 and more
            \item \textit{\textbf{Total perceived attention to environment}}
        \end{itemize}
\end{itemize}

The KPIs are reported in the following graph, with a range value of [0,4]; as we can see, in 2019 there was an improvement in all indicators, except for the perceived security. For the year 2020 the data are not available because the surveys were not carried out due to the Covid pandemic.

\begin{figure}[h!]
    \centering
    \includegraphics[width=0.9\textwidth]{Images/Social_sustainability/graph.png}
    \caption{Summary KPIs related to "Carta della Mobilità" document}
    \label{fig:kpiscdm}
\end{figure}

\subsection{Cleanliness}
Since we are living in a very particular historic period where an excellent quality of sanitary conditions is essential, we wanted to deepen the aspect of Bus Cleanliness, already partially assessed in the chapter above.

Our interest was confirmed by the fact that cleaning requirements by the PTA has changed and become stricter. The importance of the cleaning is confirmed in the service contract, which establishes penalties in case of:
\begin{itemize}
    \item failure to comply the frequency and/ or cycles in relation to the individual types of intervention (both for ordinary and extraordinary cleaning of the fleet and infrastructure and network systems open to the public)
    \item Insufficient cleaning of the bus (both for ordinary and extraordinary cleaning of the fleet and infrastructure and network systems open to the public)
\end{itemize}

The main requirements for cleanliness in today's PTA are:

\begin{itemize}
    \item daily cleaning (ordinary cleaning): it consists of removing the filth produced by the passengers, cleaning of cockpit, floor and handrails (about 10-15 min per bus)
    \item monthly cleaning (extraordinary cleaning): it is a deeper cleaning; special products must be used to remove dirt (generally it takes at least 30-40 min per bus)
    \item half yearly cleaning: vehicles are subjected to an antibacterial sanitization and disinfestation cycle 
\end{itemize}

Nowadays, due to the pandemic situation, the consortium has introduced from March 2020 sanitation interventions with disinfection and sanitization in particular of the surfaces and passenger support points. Also, every fifteen days during the periodic cleaning interventions are performed cleaning by ionization

\subsection{Mock Up Dashboard}
As per the previous chapter \ref{sec:newtech}, the aim of this is to built and visualize some mock-up dashboard pages, starting from the KPIs listed above in the previous paragraphs.

Also for this section, all pages are a mock-up also for what concerns the data present in he graphs, which may appear not close to the reality and made on purpose to talk about their benefits.

\newpage

\newpage
\begin{landscape}
\thispagestyle{empty}
\includepdf[angle=90, pagecommand={\null\enlargethispage{3\baselineskip}\vfill\captionof{figure}{Workers' Sustainability page}\label{fig:work2}}]{mockup/workers.pdf}
\end{landscape}

\subsubsection{Workers' Sustainability}
The dashboard shows the main features of workers sustainability. All the data necessary to build this type of dashboard are already available within the company and do not risk additional technologies.In order to guarantee equal treatment between men and women also in the workplace, it is also necessary to know the proportion of men and women both in the managerial area and among the traveling staff.
another indicator to monitor inequalities is the type of contract offered to men and women (permanent contract, internship, part-time contract).
Safety is also essential in the workplace. to safeguard workers' lives, in addition to strict controls on safety provisions, it is also necessary to inform and train staff. For this reason, the dashboard shows the number of hours of training followed on average by each worker. For the various indicators a trend is reported from 2018 to 2021 in order to see the improvements, as in this case, or to be able to find the critical issues and solve them.

\newpage
\begin{landscape}
\thispagestyle{empty}
\includepdf[angle=90, pagecommand={\null\enlargethispage{3\baselineskip}\vfill\captionof{figure}{Quality of Service page}\label{fig:qs}}]{mockup/qualitysocial.pdf}
\end{landscape}

\subsubsection{Quality of Service}
The data regarding the quality of the service are collected through an annual questionnaire and are therefore already available. the dashboard in particular shows the parameter concerning the comfort and cleanliness at the edge of the vehicles. it is important to note how this parameter has changed over the years and also to break down this parameter to see what are the aspects on which it is possible to improve. furthermore, through the new technologies presented above, such as the people counter, a more objective value can be given to the parameter concerning the crowding of vehicles.

At the bottom of the dashboard some data are provided regarding the number of people registered on the company's social channels and the number of applications or accesses to the official website. these aspects of the service, considered in the past less important, are useful because they are closely related to both customer loyalty and fare evasion





\newpage
\section{Green Sustainability}
\label{sec:greensus}
The management of environmental aspects in a public transport company is aimed at their continuous improvement in terms of elimination, reduction or improvement of performance.

For uniform, transparent, goal-oriented and regulation-compliant management, we have to use of a set of procedures built from initial environmental analyses. The main environmental aspects that can be identified among all the material issues are:
\begin{itemize}
    \item Emissions into the atmosphere
    \item Energy Utilization
    \item Water Waste
    \item Waste Production
    \item Use of raw materials and natural resources
    \item Soil Contamination
    \item Dust, odors, vibrations, noise and visual impact problems
\end{itemize}

The significance of impacts can be assessed according to typical environmental analysis criteria, related to the severity, size and frequency of events, taking into account their controllability, applicable legal requirements and the expectations of stakeholders - local communities, employees and public administration. 
The Sustainable Development Goals \ref{ch:Innovative} considered under this aspect are mainly five:
\begin{itemize}
    \item \textbf{n°6 – Clean Water and Sanitation}, for what concerns water waste and overall consumption
    \item \textbf{n°7 – Affordable and Clean Energy}, in particular, considering all the types of energy consumption that a public transport service company have
     \item \textbf{n°11 – Sustainable Cities and Communities}, in particular the 11.6, concerning the emission environmental impact on cities
    \item \textbf{n°12 – Responsible Consumption and Production}, for what concerns raw materials consumption and waste management
    \item \textbf{n°13 – Climate Action}, to combat climate change taking into account all the aspects related to rolling stock, atmospheric and greenhouse emissions
\end{itemize}

\begin{figure}[!ht]
    \centering
    \includegraphics[width=1\textwidth]{Images/Green Sustainability/SDGs green.png}
    \caption{Green Sustainability SDGs}
    \label{fig:grsussdg}
\end{figure}

Regarding the five areas of interest in the SDGs \ref{fig:grsussdg}, it can be built a set of KPIs according to the goals previously stated.
We have decided to analyze three big groups of aspects related to the green sustainability:

\begin{itemize}
    \item \textbf{Energy Consumption} will focus on the amount of energy consumed and how it is distributed among all the activities that a PTO has to undertake. The aim is to reduce the consumption in the most efficient way.
    \item \textbf{Emissions} is one of the major topics related to Green Mobility and it is clearly essential to assess its problem and some way to keep track of it.
    \item \textbf{Waste} will be the last aspects taken into account, where we will focus on how to optimize both residual waste and water discharges.
\end{itemize}

\subsection{Energy Consumption}
\label{subsec:enecons}
Referring at the SDG number 7 \ref{par:onuobjectives}, by 2030 companies have to significantly increase the share of renewable in the global energy mix and double the global rate of energy efficiency improvement.
Energy consumption is mostly derived from bus fuels, which can be reduced through fleet renewal, maintenance activities and improved driving style. In addition to the consumption of buses, other sources of consumption have to be monitored, such as the electricity, natural gas and heating oil at the operational sites.
Looking at the overall activities that a company can perform, here in the table \ref{tab:sourceconsumption} are listed some of the main sources of consumption:

\begin{table}[!ht]
\centering
\begin{tabular}{l|l|}
\cline{2-2}
                                            & \cellcolor{bluepoli!40}Consumption   of:                                                                                 \\ \hline
\multicolumn{1}{|l|}{Running Buses}         & \begin{tabular}[c]{@{}l@{}}Fuel (diesel and natural gas)\\ Urea (Ad Blue)\\ Antifreeze\\ Lubricants\\ Tyres\end{tabular} \\ \hline
\multicolumn{1}{|l|}{Bus Maintenance}       & \begin{tabular}[c]{@{}l@{}}Spare parts\\ Electrical energy\end{tabular}                                                  \\ \hline
\multicolumn{1}{|l|}{Vehicle Cleaning}      & \begin{tabular}[c]{@{}l@{}}Water resources\\ Electrical energy\\ Detergent consumption\end{tabular}                      \\ \hline
\multicolumn{1}{|l|}{Refuelling}            & Electrical energy                                                                                                        \\ \hline
\multicolumn{1}{|l|}{Vehicle Storage}       & \begin{tabular}[c]{@{}l@{}}Soil\\ Electrical energy\end{tabular}                                                         \\ \hline
\multicolumn{1}{|l|}{Administration}        & \begin{tabular}[c]{@{}l@{}}Methane\\ Soil   \\ Electrical energy\\ Office supplies\end{tabular}                          \\ \hline
\multicolumn{1}{|l|}{\begin{tabular}[c]{@{}l@{}}Supply of goods, \\ materials and services\end{tabular}} & Fuel, Electrical energy, Office supplies                                                               \\ \hline
\end{tabular}
\caption{Main Sources of Consumption for a LPT company}
\label{tab:sourceconsumption}
\end{table}

With that in mind we can identify a set of indicators that can help monitor all of those aspects:
\begin{itemize}
    \item Gasoline and Methane consumption in GJ per km traveled by year
    \item Electrical Energy share usage with respect to previous years
    \item Green Energy used in each activity share
    \item Electricity Consumption share for electricity and heating of offices and sites compared to total energy consumption
    \item Green or Renewable fuel share over the totality of fuel used
    \item Energy Consumption share in relation to every km traveled with respect to the previous year
\end{itemize}

\subsection{Atmosphere and GHG Emissions}
\label{subsec:ghgemissions}
Between 1990 and 2018, emissions of all greenhouse gases in Italy decreased from 516 to 428 million tonnes of CO2 equivalent, a change achieved mainly by reducing carbon dioxide emissions, which contribute 81.4 \% of the total. Emissions in 2015 were 17.1\% lower than in 1990.

\begin{figure}[!ht]
    \centering
    \includegraphics[width=1\textwidth]{Images/Green Sustainability/GHG emission.png}
    \caption{Greenhouse gas emissions from transport in the EU, by transport mode and scenario \cite{GreenhouseScenario}}
    \label{fig:ghgemissions}
\end{figure}

The energy production and transport sectors are the most important, contributing half of the national climate gas emissions. Compared to 1990, however, GHG emissions from the transport sector show a slight increase (3.2\%), while emissions from energy production and industrial installations are clearly decreasing (-23.7\% and -38.9\% respectively).

In this scenario, sustainable mobility is destined to play an important role since the increase in Local Public Transport can make a significant contribution to reducing emissions: Arriva therefore has to contribute to the transition to more energy and emission-efficient transport.
There are some actions that the company can undertake in order to commit gradually to this cause, such as:
\begin{itemize}
    \item Identify the activities that produce direct and indirect emissions both at atmospheric level and at greenhouse level
    \item Monitor and analysis of the emissions produced
    \item Implement projects and actions aimed at reducing them as well as measuring their effectiveness
    \item Raise internal and external awareness of this issue through reporting and communication to the stakeholders.
\end{itemize}

Greenhouse gas emissions can be reported in accordance with the GHG protocol (Greenhouse Gas protocol) \cite{ghgprotocol} which provides for the distinction into three categories:
\begin{description}
   \item[Scope 1] Direct emissions from the combustion of fossil fuels (diesel and natural gas) for road transport - bus fleet and car fleet - for the production of thermal energy and emissions of refrigerant gases from air conditioning systems. 
   \item[Scope 2] Emissions resulting from the production of electricity taken from the grid and consumed for the operation of plants and for lighting; the company is indirectly responsible for the emissions generated by the energy supplier for the production of the energy required. 
   \item[Scope 3] Indirect emissions, other than those from electricity consumption, which are a consequence of the company's activities and which arise from sources not owned or controlled by other organizations. The boundary of Scope 3 is defined by the organization and generally includes what can be quantified and influenced by the company.
\end{description}

As a partial solution for direct atmosphere emissions from the rolling stock, vehicles can be equipped with the innovative SCR - Selective Catalytic Reduction - system, which uses urea-based liquid, to reduce exhaust emissions in terms of particulate matter and nitrogen oxides. The system is able to reduce harmful substances in the exhaust gas of diesel vehicles by up to 80 per cent.

Wanting to quantify all of that, here below are listed some example KPIs:
\begin{itemize}
    \item Emissions of greenhouse gases (CO2, CFCs, CH4, etc.) per passenger transported or per km travelled
    \item Generic emissions (PM, VOCs, NOx, CO, etc.) per passenger transported or per km travelled
    \item Air quality standards and management plans. 
    \item Rolling Stock share with Euro5, Euro6 or EEV motorization
    \item Total km travelled by low-impact buses share
\end{itemize}

\subsection{Water Discharges and Waste Management}
\label{subsec:water}
The production of waste mainly originates from maintenance activities and bus washing. Improve the quality of water discharges from washing vehicles that may contain pollutants, such as hydrocarbons, oils and various powders is a main component of the green approach of a company Along with adequate purification plants regular inspections of the plants should be carried out, directly by the company. In the event of an emergency situation, all activities that produce water to be purified destined for the plant concerned should be suspended until the plant is operational again.

Purifiers and absorbent material will produce sludge, extracted during inspections and maintenance, which have to be stored in the plants themselves or in labeled containers at each site. The frequency of interventions has to be established on the basis of the indications in the maintenance booklets and those provided by the installer, as well as the actual use of the systems.

Storm water runoff from forecourt areas should be conveyed to plants for treatment, as required by regulations. To avoid possible contamination of runoff water the PTO can take some precautionary measures, such as:
\begin{itemize}
    \item Workshop activities have to take place in covered areas without sumps
    \item Waste storage have to takes place in covered areas and containers
    \item Storage areas that could give rise to environmental impacts on water discharges must be included in a water discharge monitoring plan
    \item Sumps and collection tanks must be inspected periodically and cleaned at least once a year
 \end{itemize}

According to all the procedures listed above, we have selected a list of KPIs that could be useful to track down this particular topic inside the company:

\begin{itemize}
    \item Liters shares of reused water
    \item Liters of storm water saved per year
    \item Tons of waste generated per year
    \item Mean time to inspection on service hours
    \item Percentage of hazardous waste recovered
    \item Tons of waste produced per 1000km traveled
\end{itemize}

\subsection{Mock Up Dashboard}
As per the previous chapter \ref{sec:newtech}, the aim of this is to built and visualize some mock-up dashboard pages, starting from the KPIs listed above in the previous paragraphs.

Also for this section, all pages are a mock-up also for what concerns the data present in he graphs, which may appear not close to the reality and made on purpose to talk about their benefits.

\newpage

\newpage
\begin{landscape}
\thispagestyle{empty}
\includepdf[angle=90, pagecommand={\null\enlargethispage{3\baselineskip}\vfill\captionof{figure}{Green Sustainability page}\label{fig:work}}]{mockup/green.pdf}
\end{landscape}

\subsubsection{Green Sustainability}
This page mainly helps the observer to look at the development of green indicators by comparing values from year to year. A much more specific set-up could also be devised, to, for example, follow the trend of emissions, energy consumed, etc., on a daily or weekly basis during the year; in this case it would also be possible to activate preventive measures in case the values are too high compared to average trends or new limits imposed by regulation.
Having a clear view of how emissions and the quantity of environmentally friendly vehicles are being managed guarantees a strategic advantage at the planning level.

In addition, data on emissions or consumption can also be useful with a much higher frequency of collection, so as to be prepared in case the PTA requires immediate changes, such as a restriction on the circulation of certain types of vehicles for a few days to reduce smog in residential areas.
 

\addtocontents{toc}{\vspace{2em}}
\newpage
\section{Maintenance \& Fleet Management}
Maintenance is a key factor in a Public Transport Service company, in particular the aim is to optimize maintenance both for what concern the active maintenance time, but also  for what concerns the reliability of all the rolling stock components.

The traditional KPIs that can generally help to assess the impact of the maintenance in a PTO can be:

\begin{description}
    \item [Unscheduled or scheduled downtime] it helps maintenance managers to analyze how successfully they have implemented maintenance strategies:
        \begin{equation}
            D (\%)= \frac{hour\:of\: unscheduled\:or\:scheduled\:downtime}{total\:time\:period}\cdot 100
        \end{equation}
        And related to that, the downtime affects the cost as:
        \begin{equation}
        D_{cost} (\$/h)= unit\: per\: hour\: \cdot\: profit\: per\: hour
        \end{equation}
    \item [Mean Time Between Failures (MTBF)]
        \begin{equation}
            MTBF = \frac{operating\:time\:hours}{number\:of\:failures}
        \end{equation}
    \item [Mean Time To Repair (MTTR)]
        \begin{equation}
            MTTR = \frac{total\:maintenance\:time}{number\:of\:repairs}
        \end{equation}
    \item [Ratio of Budgeted vs Actual Maintenance Costs] It's possible to categorize the costs into unplanned and planned maintenance, in order to find areas for improvement:
        \begin{equation}
            budgeted\:vs\:actual\:maintenance\:costs = \frac{actual\:maintenance\:cost}{budget}
        \end{equation}    
\end{description}


\subsection{Electric Bus and Infrastructure}
\label{sec:elbusinfra}
Electric buses and generally electric vehicles need their own charging infrastructure depending on their battery’s specification which is highly expensive itself. Building infrastructure for electric buses requires high financing resources and subsidiaries from PTA except if PTO wins a long-term concession. By assuming that PTO undertakes a long-term concession, the most relevant and non-negligible KPIs are gathered.

Here below we mentioned some generic important KPIs, which affect the public, end-users, and PTO commonly, in case of cost, energy, and maintenance related to the bus and its infrastructure including charging in depot and opportunity charging.

\begin{description}
    \item [Total Cost of Infrastructure] which can be divided in the following costs:
        \begin{itemize}
            \item Electric Vehicle + Battery
            \item Opportunity Charging System
            \item Charging point + Charging pole + Installation cost + Smart Charging + ICT Compliance (for charging in depot)
            \item Energy Cost
            \item Electricity Network Losses
            \item Scheduled and unscheduled repair cost of bus, battery, charger and infrastructure
        \end{itemize}
    \item [Energy Consumption] The Energy demand (KWh) can be computed as the the sum of Energy demand for single charging by opportunity charging and the Energy demand for single charging by charging overnight in depot
    
    The Energy consumption of HVAC (KWh) is instead equal to the Energy consumption due to Heating, Ventilation and Air Conditioning system
    \item [Maintenance] The total maintenance time (hours) is equal to the duration of unscheduled and scheduled repair of bus, battery, charger and infrastructure
    
    Total MTBF is the number of failure per operational hours of bus, battery and infrastructure
\end{description}


\subsection{Hydrogen bus and Infrastructure}
As mentioned in \ref{sec:elbusinfra}, it is assumed that PTO wins a long-term concession in which the chance of building the infrastructure gets higher. The most relevant KPIs related to both infrastructure and fuel cell bus were collected.

\begin{figure}[h!]
    \centering
    \includegraphics[width=0.6\textwidth]{Images/manteinance/areas_hydrogen.png}
    \caption{Areas covered in hydrogen refueling infrastructure performance assessment }
    \label{fig:areashydrogen}
\end{figure}

In terms of the performance assessment, the hydrogen refueling infrastructure consists of four areas: 

\begin{itemize}
    \item On-site hydrogen production in the HPU (Hydrogen Production Unit)
    \item Hydrogen compression, storage, and dispensing in the HRU (Hydrogen Refueling Unit)
    \item External hydrogen delivery
    \item Aspects related to the operation of the entire HRS (Hydrogen Refueling Station). 
\end{itemize}

The indicators for assessing the performance of the hydrogen refueling infrastructure and its major elements can be found in the following set of tables: 
\begin{itemize}
    \item On-site hydrogen production in the HPU (Table \ref{tab:hpu})
    \item Hydrogen compression, storage, and dispensing in the HRU (Table \ref{tab:hru})
    \item Aspects related to the operation of the entire HRS (Table \ref{tab:hrs})
\end{itemize}

% Please add the following required packages to your document preamble:
% \usepackage{multirow}
% \usepackage{graphicx}
\begin{table}[p]
\centering
\resizebox{\textwidth}{!}{%
\begin{tabular}{|l|l|l|}
\hline
\rowcolor{bluepoli!40}
\multicolumn{1}{|c|}{\textbf{KPI Name}} &
  \multicolumn{1}{c|}{\textbf{period}} &
  \multicolumn{1}{c|}{\textbf{Unit of measure}} \\ \hline
Availability of the HPU &
  \multirow{3}{*}{Monthly, annually and overall} &
  \% \\ \cline{1-1} \cline{3-3} 
\begin{tabular}[c]{@{}l@{}}Amount of hydrogen produced \\ by the HPU and \\ by eachof its electrolysers/reformers\end{tabular}  &  & \%     \\ \cline{1-1} \cline{3-3} 
\begin{tabular}[c]{@{}l@{}}Specific water consumption \\ for hydrogen production\end{tabular} &
   &
  liters/Nm3 or liters/kg \\ \hline
HPU downtime &
  \multirow{2}{*}{Annually and overall} &
  hours \\ \cline{1-1} \cline{3-3} 
\begin{tabular}[c]{@{}l@{}}Specific energy consumption \\ of the HPU and of each \\ of its electrolysers/reformers\end{tabular} &  & kWh/kg \\ \hline
\end{tabular}%
}
\caption{Performance indicators of the HPU}
\label{tab:hpu}
\end{table}
% Please add the following required packages to your document preamble:
% \usepackage{multirow}
% \usepackage{graphicx}
\begin{table}[p]
\centering
\resizebox{\textwidth}{!}{%
\begin{tabular}{|l|c|l|}
\hline
\rowcolor{bluepoli!40}
\multicolumn{1}{|c|}{\textbf{KPI Name}} &
  \textbf{period} &
  \multicolumn{1}{c|}{\textbf{Unit of measure}} \\ \hline
Availabilityof the HRU        & \multirow{2}{*}{Monthly,annually and overall} & \%     \\ \cline{1-1} \cline{3-3} 
HPUdowntime                   &                                               & hours  \\ \hline
Mean timebetween failures     & \multicolumn{1}{l|}{Annuallyand overall}      & hours  \\ \hline
Reliabilityof the HRU         & \multirow{4}{*}{Monthly,annually and overall} & \%     \\ \cline{1-1} \cline{3-3} 
\begin{tabular}[c]{@{}l@{}}Amount ofhydrogen dispensed \\ to the project buses overall and per bus\end{tabular} &
   &
  kg \\ \cline{1-1} \cline{3-3} 
Specificpower consumption HRU &                                               & kWh/kg \\ \cline{1-1} \cline{3-3} 
Speed ofdispensing            &                                               & kg/min \\ \hline
\end{tabular}%
}
\caption{Performance indicators of the HRU}
\label{tab:hru}
\end{table}
% Please add the following required packages to your document preamble:
% \usepackage{multirow}
% \usepackage{graphicx}
\begin{table}
\centering
\resizebox{\textwidth}{!}{%
\begin{tabular}{|l|c|l|}
\hline
\rowcolor{bluepoli!40}
\multicolumn{1}{|c|}{\textbf{KPI Name}} &
  \textbf{period} &
  \multicolumn{1}{c|}{\textbf{Unit of measure}} \\ \hline
\begin{tabular}[c]{@{}l@{}}Specific energy\\  consumption along \\ the on-sitehydrogen supply chain\end{tabular} &
  \multicolumn{1}{l|}{Monthly, annually and overall} &
  kWh/kg \\ \hline
Cost of hydrogen dispensed &
  \multirow{3}{*}{Annually and overall} &
  €/kg hydrogen dispensed \\ \cline{1-1} \cline{3-3} 
\begin{tabular}[c]{@{}l@{}}Number of incidents \\ affecting fuel quality\end{tabular} &
   &
  \# \\ \cline{1-1} \cline{3-3} 
Specific nitrogen consumption &
   &
  kg/tone dispensed \\ \hline
\end{tabular}%
}
\caption{Performance indicators of the HRS}
\label{tab:hrs}
\end{table}
% Please add the following required packages to your document preamble:
% \usepackage{multirow}
% \usepackage{graphicx}
\begin{table}[h!]
\centering
\resizebox{\textwidth}{!}{%
\begin{tabular}{|l|l|l|}
\hline
\rowcolor{bluepoli!40}
\multicolumn{1}{|c|}{\textbf{KPI Name}} & \multicolumn{1}{c|}{\textbf{period}}           & \multicolumn{1}{c|}{\textbf{Unit of measure}} \\ \hline
Scheduled and unscheduled repair cost & Annually and overall & €     \\ \hline
Specific fuel consumption               & \multirow{3}{*}{Monthly, annually and overall} & kg H2/ 100 km                                 \\ \cline{1-1} \cline{3-3} 
Operating hours per fuel cell system  &                      & hours \\ \cline{1-1} \cline{3-3} 
Availability of bus                   &                      & \%    \\ \hline
\end{tabular}%
}
\caption{Performance indicators of FC bus operation and refueling}
\label{tab:fc}
\end{table}

\newpage
\subsection{New Technologies and Fleet Management}
Fleet management KPIs aim at measuring the efficiency of service in many terms. But the most common benchmarks are:
\begin{itemize}
    \item boosting efficiency
    \item enhancing productivity
    \item controlling costs
\end{itemize}

Setting the specific KPIs is one aspect and meeting them is another one. With the exponential growth of technology in this era, applications and software can help to better meet KPIs by collecting real-time data. Here below are listed some fields in which that application can help:

\begin{description}
    \item [Cost Control and Budget Adherence] Because so many costs are associated with fleets, it is difficult to track expenses manually by fleet managers. Fleet managers aim to control expenses and maximize profitability: 
        \begin{itemize}
            \item Real-time cost of ownership tracking and reporting
                \begin{enumerate}
                    \item Create custom reports
                    \item Watching real-time cost per unit of distance information
                    \item Optimize vehicle usage on individual asset trends: by tracking the cost of each vehicle in real-time, you can properly respond to changing market conditions and unpredictable wear and tear.
                    \item Controlling expenses such as fuel, depreciation, fees, maintenance, administrative costs, insurance, etc. 
                \end{enumerate}
        \end{itemize}
    
    Based on the fleet reports it is possible to:
    \begin{itemize}
    \item View operational costs in real-time
        \begin{itemize}
            \item Vehicle operating costs (fuel and service)
             \item Total fleet operating cost by month
             \item Cost per mile (km or hour) trends
        \end{itemize}
    \item Track how your vehicle and equipment are being used
        \begin{itemize}
            \item Average mileage per day (utilization) by vehicle or group
            \item Vehicle assignment history across operators and vehicles
            \item History of status changes (in-shop, out-of-service, etc.)
            \item Audit trail of changes to all asset records (service, parts, issues, etc.)
        \end{itemize}
    \item 	Gain insights into how your assets are being maintained
        \begin{itemize}
            \item Service line items and cost summaries
            \item Scheduled vs. unscheduled maintenance
            \item Downtime reporting
            \item Most common service activities across your fleet
        \end{itemize}
    \item 	fuel consumption trends and efficiency
        \begin{itemize}
            \item Consumption trends by vehicle or group
            \item Fuel-ups by location
        \end{itemize}
    \item Ensure your fleet is safe and compliant
        \begin{itemize}
            \item All inspections completed and by whom
            \item View all reported defects, identify trends 
        \end{itemize}
    \end{itemize}
    \item [Maintenance Management and Downtime Prevention] Prioritizing maintenance productivity reduces downtime and maximizes efficiency. We should track issues by vehicle. Comprehensively tracking vehicle health allows you to spot recurring issues and trends across your vehicles:
    \begin{itemize}
    \item Capture issues as soon as they arise
        \begin{itemize}
            \item Mobile defect reporting
            \item Automated issue reporting
        \end{itemize}
    \item	Take action immediately
        \begin{itemize}
            \item	Coordinate with external shops electronically
            \item	Link vehicle issues to in-house Work Orders
        \end{itemize}
    \item	Collaborate in real-time
        \begin{itemize}
            \item	Mobile notifications
            \item	Interact with team members
        \end{itemize}
    \item	Track resolution and operate smarter
        \begin{itemize}
            \item	Maintain complete audit trail
        \item	Gain insight into your fleet maintenance
        \end{itemize}
    \end{itemize}
    
    To avoid recurring issues, adhering to a preventive maintenance schedule helps identify and repair issues before they compound and cause downtime:
    \begin{itemize}
    \item	Set service schedules and reminders across your fleet
        \begin{itemize}
            \item	Meter and time intervals
            \item Mobile Reminders 
        \end{itemize}
    \item	Save time, increase PM compliance, and reduce breakdowns with Service Programs
        \begin{itemize}
            \item	Keep PM schedules aligned throughout your fleet
            \item	Base Service Programs on OEM guidelines
        \end{itemize}
    \item	Predict when maintenance is due based on usage
    \end{itemize}
    \item [Optimal Vehicle Replacement Targets] Determining the best time to replace a vehicle is complex. Leveraging applications and software to monitor vehicle health and expenses can help identify optimal replacement windows. With these Analysis Tools, fleet managers can take a strategic approach to replacement and estimate replacement windows based on a variety of factors.
    \item [Fuel Costs] Fuel is one of the largest ongoing costs for fleets which can be managed but tracking fuel consumption is time-consuming if you are trying to keep up with paper receipts and manual data entry. Having drivers input fuel entries into fleet management applications, save time and allows you to view fuel costs in real-time:
    \begin{itemize}
    \item	Measure and reduce fuel costs
    \begin{itemize}
        \item 	Know your fuel economy inside and out
        \item	Understand the cost per mile for every asset
        \item	Reduce fuel theft
    \end{itemize}
    \item	Input fuel data with ease
    \begin{itemize}
         \item	Integrate your fuel card
        \item	Import your fuel data
    \end{itemize}
    \item	Simplify fuel reporting
        \begin{itemize}
            \item	View actionable trends
            \item	Share fuel insight
        \end{itemize}
    \end{itemize}
    \item [Compliance and Inspections] To maintain fleet compliance, commercial fleets must complete daily Driver Vehicle Inspection Reports (DVIR). While paper inspection forms are unorganized and inefficient, drivers can now complete with a mobile fleet management app. Drivers can upload inspection results in real-time to inform managers of vehicle issues and keep a complete record of eDVIRs to prove compliance.
\end{description}




